\section{Posłowie}
\paragraph{Co mam zapamiętać?}
\begin{itemize}
  \item kategoria to kolekcja obiektów powiązanych morfizmami,
  \item nie interesuje nas \emph{wewnętrzna struktura} obiektu (jak elementy zbioru) - ale to w jaki sposób oddziałuje z innymi obiektami (jakie morfizmy do niego wchodzą i z niego wychodzą),
  \item jako, że izomorficzne obiekty oddziałują tak samo, staramy się ich nie rozróżniać,
  \item konstrukcje nowych obiektów (jak iloczyn kartezjański czy suma przestrzeni wektorowych) mają swoje uogólnienia, składające się z obiektu i rodziny morfizmów,
  \item tak jak w kategorii różne obiekty oddziałują poprzez morfizmy, tak różne \emph{kategorie} oddziałują przez funktory. Wiele konstrukcji znanych z algebry czy geometrii różniczkowej to tak naprawdę funktory,
  \item transformacje naturalne są morfizmami między funktorami. Notacja ,,prawie takie same'' jest tak naprawdę naturalnym izomorfizmem pewnych funktorów.
\end{itemize}

\paragraph{Co mam robić teraz?}
\begin{itemize}
  \item widząc nową konstrukcję, zawsze zastanawiaj się czy nie jest ona kategoryjna (np. czy nie jest granicą). Mając relację między różnymi kategoriami (np. do każdej grupy Liego mamy przypisaną algebrę Liego) zastanawiaj się czy nie jest przypadkiem funktorialna,
  \item zapoznaj się z \emph{granicami, sprzężeniami} i \emph{lematem Yonedy}. Dobrymi książkami dla początkujących są \cite{Awodey, Baez, Leinster}.
  \item jeśli szukasz zastosowań teorii kategorii w innych dziedzinach, możesz zainteresować się algebrą przemienną \cite{AtiyahMacdonald}, algebrą homologiczną, topologią algebraiczną \cite{May} czy geometrią algebraiczną \cite{Vakil}. Możesz też przeczytać \cite{Rosetta}.
  \item jeśli lubisz logikę, oprócz \cite{Rosetta, Awodey}, możesz zacząć czytać o teorii \emph{toposów} \cite{Barr, MacLane_Moerdijk},
  \item jeśli czujesz się bardzo pewnie, spójrz na legendarną książkę \cite{MacLane} (autorem jest jeden z twórców teorii kategorii, nie jest to jednak prosta książka!),
  \item ...i pamiętaj: zawsze możesz sprawdzić nLab \cite{nLab}.
\end{itemize}

\paragraph{Sugestie i błędy} Będę wdzięczny za informację o znalezionych błędach, sugestiach czy wrażeniach z czytania (np. co było za trudne, a co zbyt łatwe, ile czasu pochłonęły te notatki).
