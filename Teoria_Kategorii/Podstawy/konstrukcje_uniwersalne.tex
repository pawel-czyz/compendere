\section{Proste konstrukcje uniwersalne}
\subsection{Izomorfizmy}
\begin{idea}
  W teorii mnogości zbiory są równe gdy mają równe elementy. Tymczasem, teoria kategorii kładzie nacisk na obiekty i morfizmy między nimi - często będziemy uznawali za ,,równe'' obiekty, które po prostu ,,zachowują się tak samo''.
\end{idea}

\begin{exmp}
  Ile jest grup dwuelementowych? Nieskończenie wiele:
  \begin{itemize}
    \item $\{0, 1\}$ z~dodawaniem modulo 2,
    \item $\{0, 2\}$ z dodawaniem modulo 4,
    \item $\{-1, 1\}$ z mnożeniem,
    \item $\{\square, \blacksquare\}$ z operacją $\square \times \square = \blacksquare \times \blacksquare = \square$ i $\square \times \blacksquare = \blacksquare\times \square = \blacksquare$,
    \item ... (wiele, wiele ,,innych'', izomorficznych grup)
  \end{itemize}
  Tak naprawdę ,,inność'' bierze się wyłącznie z użycia innych elementów - mając twierdzenie o dowolnej z tych grup, bez problemu da się je przetłumaczyć na twierdzenie o dowolnej innej. Stąd właśnie pomysł na przymykanie oka na różnice między izomorficznymi obiektami - czym jest jednak izomorfizm poza teorią grup?
\end{exmp}

\begin{defn}
  \label{defn:izomorfizm}
  \index{Izomorfizm}
  Morfizm $\morph fAB$ nazywamy \emph{izomorfizmem} jeśli istnieje $\morph gBA$ takie, że $g\circ f = \id A$ oraz $f\circ g = \id B$.
\end{defn}

\begin{exmp}
  Izomorfizmami w $\Set$ są bijekcje. Tak samo w $\textbf{FinSet}$. W $\Grp$ są to bijektywne homomomorfizmy grup. W $\Vect k$ i $\FinVect k$ - bijektywne przekształcenia liniowe. W $\Top$ izomorfizmami są homeomorfizmy.
\end{exmp}

\begin{exc}
  Pokaż, że w zbiorze częściowo uporządkowanym jedynymi izomorfizmami są identyczności.
\end{exc}

\begin{exc}
  Pokaż, że $g$ występujące w definicji \ref{defn:izomorfizm} jest unikatowe. Motywuje to pojęcie \emph{morfizmu odwrotnego do $f$} i oznaczanego przez $f^{-1}$.
\end{exc}

\begin{exmp}
  Niech $G$ będzie grupą. Możemy stworzyć kategorię o jednym obiekcie $\bullet$, której morfizmami są elementy grupy $g, h,\dots$ traktowane jak strzałki $\morpharr g\bullet\bullet, \morpharr h\bullet\bullet, \dots$ - w tej kategorii każdy morfizm jest izomorfizmem.
\end{exmp}

\begin{exc}
  Pokaż, że:
  \begin{enumerate}
    \item $\id A$ jest izomorfizmem dla dowolnego obiektu $A$,
    \item jeśli $\morph fAB$ jest izomorfizmem, to $\morph {f^{-1}}BA$ też jest izomorfizmem,
    \item jeśli $\morph fAB$ oraz $\morph gBC$ są izomorfizmami, to $g\circ f$ też jest izomorfizmem.
  \end{enumerate}
\end{exc}

\begin{defn}
  Jeśli istnieje izomorfizm z $A$ do $B$, to będziemy mówili, że \emph{$A$ i~$B$ są izomorficzne} i pisali $A\simeq B$ lub rysowali na diagramie $A\xrightarrow{~\simeq~} B$.
\end{defn}

\begin{remk}
  Nietrudno zauważyć, że ,,bycie obiektem izomorficznym'' to relacja równoważności\footnote{Po prawdzie to klasy abstrakcji mogą być kolekcjami, a nie zbiorami, więc formalnie \emph{nie} jest to relacja równoważności...}. Dlatego w teorii kategorii często przymykamy oko na różnice między izomorficznymi obiektami - ,,zachowują się tak samo''.
\end{remk}

\begin{exc}[,,Zachowują się tak samo'']
  \label{exc:izomorficzne_obiekty}
  Dla ścisłości teoriomnogościowej, pracujemy w kategorii lokalnie małej (wszystkie $\Mor AB$ są zbiorami). Załóżmy, że mamy zadany izomorfizm $\morph fA{A'}$. Wykaż, że:
  \begin{enumerate}
    \item dla dowolnego $B$, mamy bijekcję $\morph{f^*}{\Mor {A'}B}{\Mor AB}$ zadaną wzorem $f^*(\psi) = \psi \circ f$ (co jest jej odwrotnością?),
    \item ponadto jeśli diagram:
      %
      \begin{center}
      \begin{tikzcd}
        & A' \arrow[ld, "\varphi" description, bend left=25] \arrow[d, "\psi"] \\
        B \arrow[r, "k", swap] & C
      \end{tikzcd}
      \end{center}
     jest przemienny, to:
     \begin{center}
     \begin{tikzcd}
         A \ar["f", r] \ar["\varphi \circ f", d, swap] \ar["\psi \circ f", rd, bend left=25, swap]  & A' \arrow[ld, "\varphi" description, bend left=25] \arrow[d, "\psi"] \\
         B \arrow[r, "k", swap] & C
       \end{tikzcd}
    \end{center}
    %
    też jest przemienny.
  \end{enumerate}
  %
  Innymi słowy mamy bijekcję między morfizmami zaczynającymi się w $A$ oraz zaczynającymi się w $A'$. Podobnie definiując $\morph{f_* }{\Mor BA}{\Mor B{A'}}$ przez $f_*(\psi) = f\circ \psi$ mamy bijekcję między morfizmami o końcu w $A$ oraz morfizmami o końcu w $A'$.
\end{exc}

\begin{remk}
  Całkiem naturalne jest by nasze konstrukcje nie rozróżniały między izomorficznymi obiektami. Dobrą intuicją może być tutaj topologia - definiujemy niezmienniki topologiczne (np. spójność, zwartość) tak, że są takie same dla homeomorficznych przestrzeni. Podobnie, jeśli $G\simeq G'$ oraz $H\simeq H'$ są grupami (lub przestrzeniami topologicznymi), to $G\times H \simeq G'\times H'$ - nie jest to przypadek i zazwyczaj widząc taką zależność należy się spodziewać konstrukcji kategoryjnej.
\end{remk}

\subsection{Obiekt początkowy i końcowy}
\begin{defn}
  Obiekt $P$ nazywamy \emph{początkowym} jeśli dla każdego obiektu $A$ \emph{istnieje dokładnie jeden} morfizm $P\to A$. Obiekt początkowy wygląda na diagramie jak \begin{tikzcd} P \ar[r, "!", dashed] & A \end{tikzcd}.
\end{defn}

\begin{exmp}
  W $\Set$ obiektem początkowym jest tylko zbiór pusty $\varnothing$. Z kolei w $\Grp$ obiektem początkowym jest każda grupa trywialna (z~jednym elementem) - czyli jest ich bardzo dużo. Wszystkie jednak są izomorficzne. Podobnie w kategoriach $\Vect k$ i $\FinVect k$.
\end{exmp}

\begin{exc}
	Znajdź kategorię, w której \emph{nie istnieje} obiekt początkowy.
\end{exc}

\begin{exc}
	Widzimy, że obiekty początkowe zazwyczaj nie są unikatowe w sensie teoriomnogościowym. Na szczęście są ,,prawie unikatowe'' to znaczy - ,,z dokładnością do izomorfizmu''. Wykaż, że:
	\begin{itemize}
		\item jeśli $P$ i $P'$ są obiektami początkowymi, to \emph{istnieje dokładnie jeden izomorfizm} $P\to P'$,
		\item jeśli $P$ jest obiektem początkowym i $A\simeq P$, to $A$ też jest obiektem początkowym.
	\end{itemize}
\end{exc}

\begin{remk}
	Zazwyczaj jeśli $A\simeq B$, to istnieje całkiem dużo izomorfizmów $A\to B$ (ile istnieje bijekcji między dwoma zbiorami pięcioelementowymi?!). Widzimy jednak, że mając dwa obiekty początkowe, izomorfizm między nimi jest \emph{unikatowy}.
	(Jeśli masz własny ulubiony obiekt początkowy i jakieś twierdzenie o nim, oraz ja mam swój ulubiony obiekt początkowy, to nie musimy się dogadać którego izomorfizmu użyć do przeniesienia rezultatu twierdzenia na mój obiekt.)
\end{remk}

\begin{defn}
	Odwróćmy teraz kierunek strzałki - obiekt $K$ nazywamy \emph{końcowym} jeśli dla każdego obiektu $A$ \emph{istnieje dokładnie jeden} morfizm $A\to K$. Zapisując obrazkiem: \begin{tikzcd} A \ar[r, "!", dashed] & K \end{tikzcd}
\end{defn}

\begin{exmp}
  W $\Set$ obiektem końcowym jest dowolny singleton $\{\bullet\}$. W $\Grp$ obiektem końcowym jest każda grupa trywialna. W $\Vect k$ i $\FinVect k$ obiektem końcowym jest każda trywialna przestrzeń wektorowa.
\end{exmp}

\begin{exc}
  Wykaż twierdzenie:
  \begin{enumerate}
    \item jeśli $K$ i $K'$ są obiektami końcowymi to istnieje dokładnie jeden izomorfizm $K\to K'$,
    \item jeśli $A\simeq K$, to $A$ też jest obiektem końcowym.
  \end{enumerate}
\end{exc}

\begin{defn}
  Jeśli obiekt $Z$ jest jednocześnie obiektem początkowym i końcowym, to nazywamy go obiektem \emph{zerowym}. Często jest też oznaczany przez 0.
\end{defn}

\begin{exmp}
  W $\Set$ \emph{nie ma} obiektu zerowego.
\end{exmp}

\begin{exmp}
  W $\Grp$ obiektem zerowym jest grupa trywialna. Podobnie w $\Ab$. Analogicznie w $\Vect k$, $\FinVect k$ i $\Mod R$ jest to przestrzeń trywialna.
\end{exmp}

\begin{exc}
  Jeśli istnieje obiekt zerowy, to wszystkie obiekty początkowe są zerowe. Podobnie wszystkie obiekty końcowe są wtedy zerowe.
\end{exc}

\subsection{Kategoria dualna}
\index{Kategoria dualna}
\begin{defn}
  \emph{Kategorią dualną} $\op \C$ nazywamy kategorię z tymi samymi obiektami i odwróconymi strzałkami. (Czyli $\Ob \C = \Ob {\op \C}$ oraz jeśli mamy $\morpharr fAB$ w $\C$, to definiujemy strzałkę $\morpharr fBA$ w $\op \C$.)
\end{defn}

\begin{exmp}
	Obiekt $P$ jest początkowy w $\C$ wtedy i tylko wtedy gdy jest końcowy w $\op \C$. Mówimy, że obiekty końcowe i początkowe są \emph{pojęciami dualnymi}.
\end{exmp}

\begin{defn}
	Rozważmy dowolną konstrukcję wyrażoną przy pomocy diagramów. Odwracając wszystkie strzałki otrzymujemy konstrukcję \emph{dualną}. Często dodaje się przedrostek ,,ko-'' do konstrukcji dualnej.
\end{defn}

\begin{joke}
	Obiekt początkowy zwany jest też ,,ńcowym''.
\end{joke}

\begin{exc}
	Uzasadnij dlaczego izomorfizm jest pojęciem dualnym sam do siebie. (Pokaż, że jeśli $f$ jest izomorfizmem w $\C$, to jest też w $\op \C$.)
\end{exc}

\subsection{Produkt}
\begin{defn}
	\emph{Produktem} obiektów $A$ i $B$ nazywamy obiekt $P$ i morfizmy $\morpharr {\pi_A}PA$ oraz $\morpharr {\pi_B}PB$ takie, że jeśli $X$ jest dowolnym obiektem\footnote{Możemy o tym myśleć jak o konkurencie do miana produktu.} oraz $\morpharr {f_A}XA$ i $\morpharr {f_B}XB$, to \emph{istnieje dokładnie jeden} morfizm $\morph fXP$ taki, że $f_A=\pi_A\circ f$ oraz $f_B=\pi_B\circ f$.
  Jest to przedstawione na poniższym diagramie:
  %
  \begin{center}
  \begin{tikzcd}
    & X \arrow[rd, "f_B", bend left] \arrow[ld, "f_A", swap, bend right] \arrow[d, dotted, "! f"] &\\
    A & \arrow[l, "\pi_A", swap] P \arrow[r, "\pi_B"] & B
  \end{tikzcd}
  \end{center}
  %
  Często będziemy oznaczać obiekt produktu $P$ przez $A\times B$.
\end{defn}

\begin{exmp}
	Czym są produkty w $\Set$? Twierdzę, że jednym\footnote{Być może jest więcej, podobnie jak obiektów końcowych... Ciekawe czy wszystkie okażą się izomorficzne?} z dobrych produktów $A$ i $B$ jest $$(A\times B, \morpharr{\pi_A}{A\times B}A, \morpharr{\pi_B}{A\times B}B)$$
  czyli iloczyn kartezjański wraz z rzutami $\actsas{\pi_A}{(a, b)}a$ i analogicznie $\actsas{\pi_B}{(a, b)}b$.

  Weźmy teraz konkurenta do miana produktu - dowolny zbiór $X$ i funkcje $\morph {f_A}XA$ i $\morph {f_B}XB$. Zdefiniujmy:
  $$\morph f X {A\times B}, ~~\actsas{f}{x}{(f_A(x), f_B(x))}$$
  Wówczas:
	$$f_A(x) = \pi_A(f_A(x), f_B(x)) = \pi_A(f(x)) = \pi_A\circ f(x),$$
	czyli $f_A = \pi_A\circ f$ i analogicznie $f_B = \pi_B\circ f$.

  Pozostaje wykazać unikatowość $f$ - weźmy dowolną funkcję $\morph gX{A\times B}$ taką, że $f_A = \pi_A\circ g$ oraz $f_B = \pi_B \circ g$.
  Rozpatrzmy dowolne $x$ i napiszmy dla niego $g(x) = (y, z)$. Wówczas $y = (\pi_A\circ g)(x) = f_A(x)$ i analogicznie $z=f_B(x)$. Czyli $g=f$.
\end{exmp}

\begin{remk}
  Zupełnie dobrymi produktami zbiorów $A$ i $B$ są też:
  \begin{itemize}
    \item $B\times A$ z morfizmami $B\times A \ni (b, a)\mapsto a\in A$ i analogicznym $(b, a)\mapsto b$,
    \item $A\times B\times \{1\}$ z morfizmami $(a, b, 1)\mapsto a$ i $(a, b, 1)\mapsto b$,
    \item dowolny zbiór bijektywny z $A\times B$ gdy wyposaży się go w odpowiednie morfizmy.
  \end{itemize}
\end{remk}

\begin{thm}
  \label{thm:uniqueprod}
  Rozpatrzmy dwa obiekty $A$ i $B$ i przypuśćmy, że istnieje produkt $(P, \pi_A, \pi_B)$. Wówczas:
  \begin{enumerate}
    \item jeśli $(T, p_A, p_B)$ jest produktem, to \emph{istnieje unikatowy izomorfizm} $\morph iTP$ taki, że $p_A = \pi_A\circ i$ oraz $p_B = \pi_B\circ i$,
    \item jeśli mamy izomorfizm $\isomorph fQP$ to $(Q, \pi_A \circ f, \pi_B\circ f)$ też jest produktem,
    \item jeśli $A'\simeq A$ oraz $B'\simeq B$, to produkty $A' \times B'$ i $A\times B$ są izomorficzne.
  \end{enumerate}
\end{thm}

\begin{prof}
  Druga i trzecia część są proste (szczególnie jeśli zrobiło się ćwiczenie \ref{exc:izomorficzne_obiekty}), więc pokażemy tylko dowód pierwszej części:
  \begin{enumerate}
    \item weźmy produkt $(P, \pi_A, \pi_B)$ i potraktujmy jako jego konkurenta $(T, p_A, p_B)$. Mamy unikatowy morfizm $\morph iTP$ taki, że $p_A = \pi_A\circ i$ oraz $p_B = \pi_B\circ i$,
    \begin{center}
    \begin{tikzcd}
      & T \arrow[rd, "p_B", bend left] \arrow[ld, "p_A", swap, bend right] \arrow[d, dotted, "! i"] &\\
      A & \arrow[l, "\pi_A", swap] P \arrow[r, "\pi_B"] & B
    \end{tikzcd}
    \end{center}
    \item teraz weźmy $(T, p_A, p_B)$ jako produkt oraz $(P, \pi_A, \pi_B)$ jako jego konkurencję. Mamy unikatowy morfizm $\morph jPT$ taki, że $\pi_A = p_A\circ j$ oraz $\pi_B = p_B\circ j$,
    \begin{center}
    \begin{tikzcd}
      & P \arrow[rd, "\pi_B", bend left] \arrow[ld, "\pi_A", swap, bend right] \arrow[d, dotted, "! j"] &\\
      A & \arrow[l, "p_A", swap] T \arrow[r, "p_B"] & B
    \end{tikzcd}
    \end{center}
    \item teraz weźmy $(P, \pi_A, \pi_B)$ zarówno jako produkt i współzawodnika. Istnieje unikatowy morfizm $\morph mPP$ taki, że $\pi_A = \pi_A\circ m$ oraz $\pi_B=\pi_B\circ m$.
    \begin{center}
    \begin{tikzcd}
      & P \arrow[rd, "\pi_B", bend left] \arrow[ld, "\pi_A", swap, bend right] \arrow[d, dotted, "! m"] &\\
      A & \arrow[l, "\pi_A", swap] P \arrow[r, "\pi_B"] & B
    \end{tikzcd}
    \end{center}
    \item nietrudno zauważyć, że za $m$ możemy wstawić $\id P$,
    \item teraz popatrzmy na morfizm $i\circ j$. Mamy $\pi_A \circ i\circ j = p_A\circ j = \pi_A$ i~analogicznie $\pi_B\circ i\circ j = \pi_B$. Czyli za $m$ możemy wstawić $i\circ j$,
    \item $m$ jest unikatowe! Czyli $i\circ j = \id P$. Analogicznie $j\circ i=\id T$, czyli $i$ jest izomorfizmem. A skoro jest unikatową strzałką taką, że $p_A=\pi_A\circ i$ oraz $p_B=\pi_B\circ i$, to $i$ jest unikatowym izomorfizmem o zadanej własności.
  \end{enumerate}
\end{prof}

\begin{exc}
  Udowodnij drugą i trzecią część twierdzenia \ref{thm:uniqueprod}.
\end{exc}

\begin{exc}
  Czym jest produkt dwóch grup? A przestrzeni topologicznych?
\end{exc}

\begin{exc}
  Podaj przykład pokazujący, że produkt nie zawsze istnieje.
\end{exc}

\begin{exc}[Pozornie inna definicja produktu]
  \label{exc:produkt_poczatkowy}
  {
  \newcommand{\Prod}{\text{Prod}(A, B)}
  Rozpatrzmy obiekty $A$ i $B$ kategorii $\C$. Definiujemy kategorię ,,obiektów produktopodobnych'' $\Prod$ w następujący sposób:
  \begin{itemize}
    \item obiektami są trójki $(X, f, g)$ gdzie $X$ jest obiektem $\C$, a $\morpharr fXA$ i~$\morpharr gXB$ są morfizmami $\C$,
    \item morfizmem między obiektami $(X, f, g)$ oraz $(Y, p, q)$ będziemy nazywać morfizm $\morpharr mXY$ taki, że $f=p\circ m$ oraz $g=q\circ m$.
  \end{itemize}

  \begin{enumerate}
    \item Po pierwsze upewnij się, że to jest kategoria. (Czym jest składanie morfizmów? Czym są morfizmy identycznościowe? Warto narysować diagram.)
    \item Pokaż, że dowolny produkt $(A\times B, \pi_A, \pi_B)$ jest obiektem końcowym w $\Prod$.
  \end{enumerate}
  Innymi słowy \emph{produkty to obiekty końcowe} (pewnego rodzaju).
  }
\end{exc}

\begin{exc}
  Zdefiniuj produkt dowolnej rodziny obiektów $(A_i)_{i\in I}$.
\end{exc}

\begin{exc}[Zabawy ze zbiorami częściowo uporządkowanymi]
  \begin{enumerate}
    \item Pokaż, że w $\mathbb Z^+$ uporządkowanym przez podzielność (przykład \ref{exmp:dzielenie}) produktem liczb $n$ i $m$ jest $\mathrm{nwd}(n, m)$.
    \item Niech $\mathcal P(S)$ będzie zbiorem potęgowym $S$. Pokaż, że produktem zbiorów $A_1, A_2, \dots, A_n\subseteq S$ jest $A_1\cap A_2\cap \dots \cap A_n$.
    \item Pokaż ogólniejszy fakt - w zbiorze częściowo uporządkowanym $S$, produktem elementów w podzbiorze $X\subseteq X$ jest infimum $X$ (jeśli istnieje). (Element $\inf X$ jest zdefiniowany jako największe ograniczenie dolne $X$, to znaczy $\inf X\le x$ dla wszystkich $x\in X$ oraz jeśli $i\le x$ dla wszystkich $x\in X$, to $i\le \inf X$.)
  \end{enumerate}
\end{exc}

\begin{remk}[O łączności]
  Nie jest trudno wykazać (choć to dość żmudne), że jeśli $(A\times B, \pi_A, \pi_B)$ jest produktem $A$ i $B$, a $((A\times B)\times C,\, p_{A\times B},\, p_C)$ jest produktem $A\times B$ i~$C$, to obiekt $(A\times B)\times C$ wraz z morfizmami $\pi_A\circ p_{A\times B},\, \pi_B\circ p_{A\times B},\, p_C$
  jest produktem $A$, $B$ i $C$. Ma to dwie istotne konsekwencje:
  \begin{enumerate}
    \item jeśli wiemy, że w danej kategorii istnieje produkt każdych dwóch obiektów, to istnieje też produkt każdej ich skończonej liczby,
    \item chociaż często produkt nie jest łączny (np. w teorii mnogości $(A\times B)\times C\neq A\times (B\times C)$), to jednak istnieją unikatowe ,,ładne'' izomorfizmy, utożsamiające różne produkty (jak $((a, b), c)\mapsto (a, (b, c))$). Za pojęciem ,,obiekty są prawie takie same'' stoją izomorfizmy naturalne (pomysł \ref{idea:prawie_takie_same}).
  \end{enumerate}
\end{remk}

\subsection{Koprodukt}
\begin{defn}
  \emph{Koprodukt} jest pojęciem dualnym do produktu, to znaczy - weźmy obiekty $A$ i $B$. Koproduktem nazywamy obiekt $A+B$ oraz morfizmy $\morph{\sigma_A} A{A+B}$ i $\morph{\sigma_B}B{A+B}$, takie, że jeśli $(K, \morpharr {k_A}AK, \morpharr {k_B}BK)$, to istnieje unikatowy morfizm $\morpharr k{A+B}K$ taki, że $k_A = k \circ \sigma_A$ i $k_B = k\circ \sigma_B$.
\end{defn}

\begin{exc}[Własności koproduktu]
  Znając własności produktu i odwracając strzałki, dostajemy analogiczne własności koproduktu:
  \begin{enumerate}
    \item narysuj diagram definiujący koprodukt (wystarczy odwrócić strzałki w~definicji produktu),
    \item zauważ, że jeśli mamy dwa koprodukty $(A+B, \sigma_A, \sigma_B)$ oraz $(K, k_A, k_B)$, to istnieje unikatowy izomorfizm $\morph i{A+B}K$ taki, że $k_A=i\circ \sigma_A$ oraz $k_B=i\circ\sigma_B$,
    \item zinterpretuj koprodukt jako obiekt początkowy w jakiejś kategorii (pomocne może być ćwiczenie \ref{exc:produkt_poczatkowy}).
  \end{enumerate}
\end{exc}

\begin{exmp}
  Koproduktem dwóch zbiorów $A$ i $B$ (lub przestrzeni topologicznych) jest ich suma rozłączna $A\coprod B$, wraz z inkluzjami $A\hookrightarrow A\coprod B$, $B\hookrightarrow A\coprod B$.
\end{exmp}

\begin{exmp}
  Koproduktem dwóch przestrzeni wektorowych (ogólniej - modułów, czyli także grup abelowych) jest ich suma prosta. Jako, że koprodukt i~produkt w tym wypadku są tym samym, nazywamy je czasami \emph{biproduktem}. Natomiast koprodukt nieskończonej rodziny przestrzeni wektorowych (czy modułów) już nie jest tym samym co produkt.
\end{exmp}

\begin{exmp}
  Koproduktem dwóch grup jest iloczyn wolny (ta konstrukcja nie jest podstawowa, mogła być pominięta na kursie algebry).
\end{exmp}

\begin{exc}
  Czym jest koprodukt w $\mathbb Z^+$ uporządkowanym przez podzielność? A w zbiorze potęgowym uporządkowanym przez inkluzję?
\end{exc}

\subsection{Ekwalizator}
\begin{defn}
  Rozważmy równoległe strzałki \begin{tikzcd} X \ar[r, "f", shift left=.75ex] \ar[r, "g", shift right=.75ex, swap] & Y \end{tikzcd} (to jest - dwie strzałki o wspólnej dziedzinie i przeciwdziedzinie. Niekoniecznie $f=g$, więc diagram nie jest do końca przemienny...). \emph{Ekwalizatorem} nazywamy obiekt $E$ i morfizm $\morph eEX$ taki, że:
  \begin{itemize}
    \item $f\circ e = g\circ e$,
    \item jeśli $Q$ i $\morph qEX$ spełniają warunek $f\circ q = g\circ q$, to istnieje unikatowy morfizm $\morph kQE$ taki, że $q=e\circ k$.
  \end{itemize}
  %
  Definicję tę przedstawia poniższy diagram prawie (nie wymagamy $f=g$) przemienny:
  \begin{center}
  \begin{tikzcd}
    E \ar[r, "e"] & X \ar[r, "f", shift left=.75ex] \ar[r, "g", shift right=.75ex, swap] & Y\\
    Q \ar[ru, "q"] \ar[u, "! k", dotted]
  \end{tikzcd}
  \end{center}
\end{defn}

\begin{exc}
  Zinterpretuj stwierdzenie ,,ekwalizator jest unikatowy z dokładnością do unikatowego izomorfizmu''.
\end{exc}

\begin{exc}[Ale dlaczego ,,ekwalizator''?]
  Pokaż, że w $\Set$ ekwalizatorem \begin{tikzcd} X \ar[r, "f", shift left=.75ex] \ar[r, "g", shift right=.75ex, swap] & Y \end{tikzcd} jest zbiór:
    $$E = \{e\in X : f(x) = g(x) \}$$
  wraz z inkluzją $E\hookrightarrow X$.
\end{exc}

\begin{exc}
  Wybierz swoją ulubioną kategorię spośród $\Vect k, \Mod R, \Ab$ i pokaż, że ekwalizatorem morfizmów \begin{tikzcd} X \ar[r, "f", shift left=.75ex] \ar[r, "g", shift right=.75ex, swap] & Y \end{tikzcd} jest jądro ich różnicy (to jest $\ker (f-g)$) wraz z inkluzją.
\end{exc}

\begin{remk}[\starred]
    W $\Mod R$ (czyli też $\Vect k$ i $\Ab$) istnieje obiekt zerowy $0$. Czyli istnieją morfizmy $X\to 0$ i $0\to Y$. Składając je otrzymujemy morfizm zerowy $\morpharr 0 X Y$. W ten sposób otrzymujemy kategoryjną własność jądra $f$ jako ekwalizatora \begin{tikzcd} X \ar[r, "f", shift left=.75ex] \ar[r, "0", shift right=.75ex, swap] & Y \end{tikzcd}.
\end{remk}

\begin{remk}[\starred]
  Ekwalizator (i powyższa interpretacja jako jądro różnicy) są przydatne do zdefiniowania tzw. snopów.
\end{remk}

\begin{exc}
  Zdefiniuj obiekt dualny - koekwalizator. (Jest rzadziej spotykany - w $\Set$ odpowiada dzieleniu przez pewną relację równoważności, a w $\Mod R$ staje się \emph{kojądrem} przekształcenia.)
\end{exc}

% \subsection{Granice i kogranice}
% \begin{idea}
%   W powyższych przykładach powtarza się motyw:
%   \begin{itemize}
%     \item rysujemy diagram składający się z obiektów i strzałek
%     \item szukamy obiektu i strzałek (po jednej dla obiektu), które są ,,uniwersalne'' w pewnym sensie - mając innego kandydata na konstrukcję, jesteśmy w stanie znaleźć unikatowy morfizm między nimi.
%   \end{itemize}
%   Ta idea prowadzi nas do ogólnej definicji granicy (i konstrukcji dualnej - kogranicy).
% \end{idea}
%
% \begin{exmp}
%   Poznaliśmy następujące granice:
%   \begin{enumerate}
%     \item diagram $A~B$ (brak morfizmów) -
%     \item diagram pusty $\varnothing$ - szukamy obiektu $P$
%   \end{enumerate}
% \end{exmp}

\begin{remk}
  Konstrukcje, które wykonywaliśmy nazywane są konstrukcjami uniwersalnymi - zadajemy za pomocą diagramu pewną własność i otrzymujemy obiekt (i morfizmy) odpowiadające obiektowi końcowemu (produkt, ekwalizator) pewnej kategorii lub początkowemu (koprodukt, koekwalizator). W przypadku obiektów końcowych konstrukcje takie nazywamy \emph{granicami}, w przypadku początkowych - \emph{kogranicami}. Dają one zunifikowany pogląd na wiele konstrukcji popularnych w algebrze czy geometrii.
\end{remk}
