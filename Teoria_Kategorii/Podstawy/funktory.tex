\section{Funktory}
\begin{idea}
  Tak jak morfizmy są odwzorowaniami między obiektami, tak \emph{funktory} są \emph{odwzorowaniami między kategoriami}. Pozwala to na przenoszenie twierdzeń i problemów z jednej kategorii do innej.
\end{idea}

\begin{defn}[Funktor kowariantny]
  Niech $\C$ i $\D$ będą kategoriami. \emph{Funktorem kowariantnym} $\morph F\C\D$ nazywamy przyporządkowanie o własnościach:
  \begin{enumerate}
    \item Dla każdego obiektu $A\in \Ob\C$ mamy pewien $F(A)\in \Ob\D$,
    \item Dla każdego morfizmu $\morpharr fAB$, mamy morfizm $\morpharr {F(f)}{F(A)}{F(B)}$.
  \end{enumerate}
 Przy czym wymagamy aby:
  \begin{enumerate}
    \item $F(g\circ f) = F(g)\circ F(f)$ (złożenia są zachowywane),
    \item $F(\id A) = \id{F(A)}$ (identyczności są zachowywane).
  \end{enumerate}
\end{defn}

\begin{exmp}[Funktor stały]
  Niech $D\in \Ob \D$ będzie ustalonym obiektem. Definiujemy:
  \begin{enumerate}
    \item $S(A)=D$
    \item $S(\morpharr fAB) = \id D$
  \end{enumerate}
  Sprawdźmy wszystkie własności:
  \begin{enumerate}
    \item Strzałka $\morph fAB$ przechodzi na strzałkę $\morph{\id D}{S(A)}{S(B)}$,
    \item Oczywiście $S(\id A)=\id {D} = \id {S(A)}$,
    \item $S(g)\circ S(f) = \id D\circ \id D = \id D = S(g\circ f)$.
  \end{enumerate}
  Czyli jest to funktor.
\end{exmp}

\begin{exmp}[Funktor identycznościowy]
  Funktor identycznościowy $\morph {\id \C}\C \C$ to następujące przekształcenie:
  \begin{enumerate}
    \item $\id \C(A) = A$,
    \item $\id \C(\morpharr fAB) = \morpharr fAB$
  \end{enumerate}
  Nietrudno zauważyć, że zachowuje złożenia i identyczności.
\end{exmp}

\begin{exc}
  \label{exc:group_forgetful}
  Pokaż, że odwzorowanie ,,zapominalskie'' $\morph U\Grp \Set$, które:
  \begin{enumerate}
    \item Grupie $G$ przyporządkowuje jej nośnik\footnote{Nośnik to zbiór elementów -- formalnie grupa $G$ to pewien zbiór $U(G)$ i działanie $\circ$ o odpowiednich własnościach.} $U(G)$,
    \item Homomomorfizmowi $G\to H$ przyporządkowuje siebie $U(f)=f$ (homomomorfizmy to przecież funkcje),
  \end{enumerate}
  jest funktorem\footnote{Aby być w pełni poprawnym, to należałoby ,,rozpakować'' homomorfizm z trójki $(G, f, H)$ i ,,zapakować'' go w zbiory, ale nie przejmowałbym się tym szczegółem zanadto. Analogicznym szczegółem jest to, że piszemy $G$ zamiast $(G, \circ)$, bo teoriomnogościowo grupa to zbiór \emph{z działaniem}.}.
\end{exc}

\begin{exc}
  Pokaż, że odwzorowanie $\morph{\mathcal P}\Set\Set$ jest funktorem, gdzie:
  \begin{itemize}
    \item $\mathcal P(A) = \text{zbiór potęgowy $A$}$,
    \item $\mathcal P(f) = \mathrm{Im}\,f = f(A)\subseteq B$, jeśli $\morph fAB$.
  \end{itemize}
\end{exc}

\begin{exmp}[\starred Iloczyn tensorowy]
  Jeśli $V$ jest ustaloną przestrzenią wektorową nad ciałem $k$, to odwzorowanie $W\mapsto W\otimes_k V$, $f\mapsto f\otimes \id V$ jest funktorem $\morph{\bullet \otimes_k V}{\Vect k}{\Vect k}$.
\end{exmp}

\begin{exmp}[\starred Dla topologów algebraicznych]
  W topologii algebraicznej bierzemy przestrzeń topologiczną $X$ oraz punkt $x_0$ i przypisujemy tzw. grupę fundamentalną $\pi_1(X, x_0)$ w punkcie $x_0$. Funkcjom ciągłym przenoszącym wyróżniony punkt na inny wyróżniony punkt, mamy ponadto odpowiadający homomomorfizm grup fundamentalnych, zachowujący złożenia i identyczności. Czyli grupa fundamentalna to tak naprawdę funktor $\morph{\pi_1}{\Top_*}\Grp$ (z kategorii przestrzeni topologicznych z wyróżnionym punktem do kategorii grup).
\end{exmp}

% \begin{exmp}[\starred Dla geometrów różniczkowych]
%   Funktor styczny przypisuje rozmaitości $M$ jej wiązkę styczną $TM$. Funkcji gładkiej $\morph fMN$ przypisuje odwzorowanie styczne: $Tf(v_p) = df|_pv_p\in T_{f(p)}N$.
%   Pochodną funkcji identycznościowej jest oczywiście odwzorowanie identycznościowe --- $(T\id M)(v_p) = d(\id M)|_pv_p = \id {T_pM} v_p$. Podobnie twierdzenie o pochodnej funkcji złożonej ---$T(g\circ f)v_p = d(g\circ f)|_p v_p = dg|_{f(p)}\circ df|_p v_p = (Tg\circ Tf)v_p$ --- pokazuje, że złożenie też się dobrze przenosi.
% \end{exmp}

% \begin{exmp}[\starred Dla koneserów teorii Liego]
%   Jeśli $G$ jest grupą Liego, to możemy przyporządkować jej algebrę Liego $\mathfrak g$ (przestrzeń lewoniezmienniczych pól wektorowych, izomorficzną z przestrzenią styczną obliczoną w identyczności $e\in G$). Ponadto jeśli $\morph \varphi GH$ jest homomomorfizmem grup Liego, to pochodna $d\varphi|_e$ jest homomorfizmem algebr Liego $\mathfrak g\to \mathfrak h$. Nietrudno pokazać, że odwzorowanie to przenosi identyczności na identyczności i zachowuje złożenia.
% \end{exmp}

\begin{remk}
  Funktor \emph{nie musi} ,,dobrze'' przenosić konstrukcji uniwersalnych.

  Rozważmy ,,zapominalski'' funktor $\morph U{\Vect k} \Set$ zamieniający przestrzenie wektorowe na ich nośniki (zbiory wektorów). Jeśli $V$ i $W$ są przestrzeniami wektorowymi, to nośnikiem ich koproduktu (sumy) jest iloczyn kartezjański: $U(V+W) = U(V)\times U(W)$, a koproduktem zbiorów $U(V)$ i~$U(W)$ jest suma rozłączna $U(V)\coprod U(W)$, czyli coś zupełnie innego!
\end{remk}

\begin{exc}
  \label{exc:izo_preserved}
  Pokaż, że funktory zachowują izomorfizmy -- jeśli $\morph fAB$ jest izomorfizmem, to $\morph {F(f)}{F(A)}{F(B)}$ też jest izomorfizmem.
\end{exc}

\begin{remk}
  Jeśli chcemy wykazać, że obiekty $A$ i $B$ nie są izomorficzne, wystarczy znaleźć funktor $F$ taki, że $F(A)$ i $F(B)$ \emph{nie} są izomorficzne. Grupa fundamentalna czy teorie homologii to właśnie sposoby rozróżniania obiektow topologicznych przez zamianę ich na obiekty algebraiczne.
\end{remk}

\begin{defn}[Funktor kontrawariantny]
  \emph{Funktorem kontrawariantnym} $\morph F\C\D$ nazywamy funktor kowariantny $\morph F {\op \C}{\D}$. Innymi słowy:
  \begin{enumerate}
    \item Dla każdego obiektu $A\in \Ob\C$ mamy pewien $F(A)\in \Ob\D$,
    \item Dla morfizmu $\morpharr fAB$, mamy morfizm $\morpharr {F(f)}{F(B)}{F(A)}$.
  \end{enumerate}
 Przy czym wymagamy aby:
  \begin{enumerate}
    \item $F(g\circ f) = F(f)\circ F(g)$ (złożenia są odwracane),
    \item $F(\id A) = \id{F(A)}$ (identyczności są zachowywane).
  \end{enumerate}
\end{defn}

\begin{exc}[\starred Dla topologów]
  Pokaż, że odwzorowanie $\morph{\mathcal P}\Set\Set$ jest funktorem kontrawariantnym, gdzie:
  \begin{itemize}
    \item $\mathcal P(A) = \text{zbiór potęgowy $A$}$,
    \item $\mathcal P(f) = f^{-1}(A) \subseteq B$, jeśli $\morph fAB$.
  \end{itemize}
\end{exc}

\begin{exc}[$\mathrm{Hom}$-funktory]
  Rozważmy kategorię $\C$ i obiekt $A\in \Ob \C$. Stwórzmy odwzorowanie $\morph {\Mor A \bullet} \C \Set$ dane wzorem:
  \begin{itemize}
    \item $\Mor A \bullet(B) = \Mor AB$ (i tak też będziemy pisać --- kropeczka ma symbolizować puste miejsce na argument),
    \item $\Mor A f = f_*$, gdzie jeśli $\morph fB{B'}$, to $\morph{f_*}{\Mor AB}{\Mor A{B'}}$ jest dane wzorem $f_*(\varphi) = f\circ \varphi$.
  \end{itemize}
  W ramach zadania:
  \begin{enumerate}
    \item Pokaż, że jest to funktor kowariantny,
    \item Zdefiniuj \emph{kontrawariantny} funktor $\Mor \bullet A$. (Podpowiedź: już się z~nim spotkaliśmy --- możesz motywować się ćwiczeniem \ref{exc:izomorficzne_obiekty}.)
  \end{enumerate}
  %
  Te funktory są szczególnie ważne w algebrze homologicznej oraz w ważnym rezultacie teorii kategorii -- lemacie Yonedy.
\end{exc}

% \begin{exmp}[\starred Spektrum pierścienia\cite{AtiyahMacdonald}]
% {
%   \newcommand{\CRing}{\textbf{CRing}}
%   \newcommand{\Spec}[1]{\mathrm{Spec}\,#1}
%   W tym przykładzie zajmujemy się $\CRing$, czyli kategorią pierścieni przemiennych z jedynką.
%   Niech $A$ będzie takowym pierścieniem. Możemy mu przypisać zbiór ideałów pierwszych $\Spec A$. Jeśli $\mathfrak p\subseteq B$ jest ideałem pierwszym i $\morph \varphi AB$ jest homomorfizmem, to $\varphi^{-1}(\mathfrak p)$ jest ideałem pierwszym w $A$. Można więc zdefiniować funkcję $\morph{\varphi^*}{\Spec B}{\Spec A}$. Można pokazać, że tak zdefiniowane odwzorowanie jest funktorem kontrawariantnym z $\CRing$ do $\Set$. (A tak naprawdę to w~$\Top$, gdy wyposaży się spektrum w topologię Zariskiego.)
% }
% \end{exmp}
