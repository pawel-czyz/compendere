\section{Sprzężenia}
\begin{noth}[Przestrzeń wolna]
  Niech $S$ będzie zbiorem, a $k$ ustalonym ciałem. Weźmy funkcję $\morph f Sk$ taką, że $f(v) = 0$ dla prawie wszystkich $v$ (czyli istnieje tylko skończenie wiele wektorów $v$ takich, że $f(v)\neq 0$).

  Nietrudno zauważyć, że funkcje takie tworzą przestrzeń wektorową.
\end{noth}

\begin{exmp}
  Rozważmy kategorie $\Vect k$ oraz $\Set$. Możemy wprowadzić dwa funktory:
  \begin{itemize}
    \item ,,funktor zapominalski'' $\morph{U}{\Vect k}{\Set}$, który przestrzeni wektorowej przyporządkowuje jej nośnik (zbiór wektorów) i nie zmienia funkcji (bo funkcje liniowe to przecież funkcje!),
    \item ,,wolny funktor'' $\morph{F}{\Set}{\Vect k}$. Mając dany zbiór $S$ przekształca go na wolną przestrzeń wektorową.
  \end{itemize}
\end{exmp}
