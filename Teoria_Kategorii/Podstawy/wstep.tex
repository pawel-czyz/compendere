\section{Czym są kategorie?}
\begin{remk}
  Zbiory intuicyjnie rozumiane są jako pewne kolekcje czy też woreczki, które mogą zawierać jakieś przedmioty. Taka intuicja może jednak prowadzić na manowce - nie istnieje zbiór wszystkich zbiorów. Tymczasem zwrot ,,kolekcja wszystkich zbiorów'' jest intuicyjnie zrozumiały, mimo, że nie może być zbiorem.

  Bardzo niefrasobliwie, nie będziemy przejmować się aksjomatyką ,,kolekcji'', nasze notatki nie będą więc ścisłe w sensie teoriomnogościowym. Sugerujemy przymknąć na to oko - zamiast koncentrować się na technikaliach, zajmiemy się tym co jest pojęciowo ważne. Niektorzy też badają aksjomatyki, w której wychodzi się od pojęcia kategorii zamiast od zbioru. (Czytelnika o~teoriomnogościowym nastawieniu do życia, zachęcamy do konsultacji z \emph{uniwersami Grothendiecka} lub teoriami \emph{klas}, np. aksjomatyką NBG)
\end{remk}

\begin{idea}
  Zbiór to kolekcja elementów. \emph{Kategoria} to kolekcja obiektów i~strzałek, które łączą różne obiekty. (Np. kolekcja wszystkich zbiorów połączona funkcjami.)
  Innymi słowy, zamiast odpowiadać na teoriomnogościowe pytanie o pojedynczy przedmiot: ,,jakie elementy ma ten zbiór?'' kładziemy nacisk na całościowe spojrzenie: ,,w jaki sposób ten obiekt jest oddziałuje z innymi?''.
\end{idea}

\begin{defn}
  \index{Kategoria}
  Będziemy nazywać $\C$ \emph{kategorią} jeśli:
    \begin{itemize}
      \item mamy pewną kolekcję \emph{obiektów} $A, B, C, \dots$. Będziemy oznaczać tę kolekcję przez $\Ob \C$,
      \item mamy pewną kolekcję \emph{strzałek} $f, g, h\dots$. Strzałki będziemy nazywać też \emph{morfizmami},
      \item każda strzałka $f$ łączy dokładnie dwa określone obiekty - \emph{dziedzinę} z \emph{przeciwdziedziną}. Ten fakt oznaczamy przez $\morph fAB$ lub $\morpharr fAB$,
      \item strzałki można \emph{składać} w następujący sposób:
        \begin{itemize}
          \item jeśli $\morpharr fAB$ oraz $\morpharr gBC$, to istnieje strzałka $\morpharr {g\circ f}AC$. Będziemy czasami pisać $gf$ zamiast $g\circ f$,
          \item jeśli $\morpharr fAB$, $\morpharr gBC$ oraz $\morpharr h CD$, to zachodzi $(h\circ g)\circ f = h\circ (g\circ f)$ (złożenie jest \emph{łączne}),
          \item dla każdego obiektu $A$ istnieje strzałka $\morph{\id A}AA$ taka, że dla każdej $\morph fAB$ to $\id B \circ f = f = f \circ \id A$ (istnieją \emph{morfizmy identycznościowe}).
        \end{itemize}
    \end{itemize}
\end{defn}

\begin{exmp}
  \label{exmp:set}
  Kategorią $\Set$ nazywamy kolekcję wszystkich zbiorów, w której strzałka $\morpharr fAB$ to po prostu trójka $(A, f, B)$, gdzie $f$ jest funkcją ze zbioru $A$ w zbiór $B$. Złożenie strzałek określamy przez złożenie funkcji, a strzałki identycznościowe otrzymujemy z funkcji identycznościowych.

  Pojawia się pytanie - dlaczego definiujemy strzałkę jako \emph{trójkę} zamiast po prostu funkcję? Teoriomnogościowo funkcje $\morph f{\mathbb R}{\mathbb R}$ oraz $\morph {f'}{\mathbb R}{\mathbb Z}$ zadane wzorem $x\mapsto 1$ są tym samym - relacją $\mathbb R\times \{1\}$. W teorii kategorii jednak chcemy rozróżniać $f$ i $f'$, dlatego ręcznie dodajemy informację o dziedzinie i~przeciwdziedzinie.

  Jako, że ręczne dodanie informacji jest proste, to popularnie (i nie do końca ściśle) mówi się, że $\Set$ to kategoria w której obiektami są zbiory, a~morfizmami są funkcje.
\end{exmp}

\begin{exmp}[Inne ,,standardowe'' kategorie]
  \label{exmp:standard}
  Podobnie jak w powyższym przykładzie, mamy kategorie (uwaga - tak teoriomnogościowo to morfizmami są trójki, a nie funkcje):
  \begin{enumerate}
    \item $\FinSet$ - obiektami są zbiory skończone, a morfizmami - funkcje,
    \item $\Grp$ - obiektami są grupy, a morfizmami homomomorfizmy grup (składanie homomomorfizmów jest łączne, identycznością jest homomomorfizm identycznościowy),
    \item $\Top$ - obiektami są przestrzenie topologiczne, a morfizmami funkcje ciągłe,
    \item $\Mod R$ - obiektami są lewostronne moduły nad pierścieniem $R$, a morfizmami funkcje $R$-liniowe (analogicznie prawe moduły tworzą kategorię),
    \item $\Vect k$ - obiektami są przestrzenie wektorowe nad ciałem $k$, a morfizmami są funkcje liniowe (jest to szczególny przypadek kategorii modułów - $R=k$),
    \item $\Ab$ - obiektami są grupy abelowe, a morfizmami homomomorfizmy grup (to też szczególny przypadek kategorii modułów - $R=\mathbb Z$),
    \item $\FinVect k$ - obiektami są przestrzenie wektorowe skończonego wymiaru nad ciałem $k$, a morfizmami są funkcje liniowe.
  \end{enumerate}
  %
  Podobnych przykładów można wymieniać bez liku (monoidy, pierścienie, ciała, rozmaitości różniczkowe...). Często obiekty to po prostu zbiory wyposażone w dodatkową strukturę, a morfizmami są funkcje zachowującą tę strukturę. Warto pamiętać, że \emph{kategorie obejmują także inne przykłady}.
\end{exmp}

\begin{nott}
  $\Mor AB$ to kolekcja wszystkich strzałek $\morpharr fAB$. W literaturze popularne są także oznaczenia\footnote{Dlaczego Hom, a nie Mor? Zapewne dlatego, że np. w teorii grup czy pierścieni, na strzałki mówi się \emph{homomorfizmy}.} $\bareMor_\C(A, B)$ lub $\mathrm{Mor}(A, B)$.
\end{nott}

\begin{defn}
Przyjmujemy oznaczenia:
\begin{enumerate}
  \item kategoria, w których zarówno kolekcja obiektów jak i kolekcja strzałek są zbiorami, nazywana jest \emph{małą},
  \item kategoria, w której wszystkie kolekcje $\Mor AB$ są zbiorami jest nazywana \emph{lokalnie małą}.
\end{enumerate}
\end{defn}

\begin{exmp}
  Kategorie z przykładu \ref{exmp:standard} są lokalnie małe, lecz nie są małe.
\end{exmp}

\begin{remk}[Uwaga techniczna, można pominąć]
  \label{remk:rozlaczne}
  Warto zauważyć, że $\Mor AB$ oraz $\Mor{A'}{B'}$ są albo identyczne (gdy $A=A'$ i $B=B'$) albo rozłączne (w przeciwnym przypadku).
  To wymaganie jest czysto techniczne (każda strzałka musi łączyć dokładnie dwa obiekty) - mając zbiory $\bareMor'(A, B)$, które nie są parami rozłączne możemy zastąpić je przez $\Mor AB =\{A\}\times \bareMor'(A, B) \times \{B\}$, to znaczy zamiast rozpatrywać $\morph fAB$ wystarczy rozpatrywać trójki uporządkowane $(A, f, B)$, tak samo jak w przykładzie \ref{exmp:set} o kategorii $\Set$.
\end{remk}

\begin{exmp}
  \label{exmp:natur}
  {\newcommand{\naturalnum}{\mathbb N}
  Nie każda kategoria wygląda jak ,,zbiory wzbogacone o jakąś strukturę'' i ,,pewne funkcje''. Jako kolekcję obiektów weźmy liczby naturalne. Będziemy rysować strzałkę z $m$ do $n$ wtedy i tylko wtedy gdy $m\le n$. Nietrudno zauważyć, że jest to kategoria:
  \begin{itemize}
    \item jeśli $m\le n$ oraz $n\le p$, to $m\le p$, więc określamy złożenie strzałek $\morpharr{}mn$ i $\morpharr{}np$ jako unikalną strzałkę $\morpharr{} mp$,
    \item między każdymi dwiema liczbami istnieje co najwyżej jedna strzałka, więc o problemach z łącznością nie ma mowy,
    \item strzałkami identycznościowymi są $\id n = (n\le n)$, które składają się poprawnie.
  \end{itemize}
  %
  (Jeśli takie określenie kategorii budzi dyskomfort, warto przyjąć:
  $$
  \Mor mn =
  \begin{cases}
    \{ (m, n) \} \text{ jeśli $m\le n$}\\
    \varnothing \text{ w przeciwnym przypadku}
  \end{cases}
  $$
  wraz z działaniem $(n, p)\circ (m, n) = (m, p)$.)
  }
  Taką kategorię można też narysować:
  \begin{center}
  \begin{tikzcd}
    1 \ar[r] \ar[rr, bend left] \ar[rrr, bend left=50] & 2 \ar[r] \ar[rr, bend left] & 3 \ar[r] & \dots
  \end{tikzcd}
  \end{center}
  Na obrazku pominięto morfizmy identycznościowe (które wyglądają jak pętelki).
\end{exmp}

\begin{exmp}[\starred Dla topologów]
  Można też definiować tak zwaną \emph{naiwną kategorię homotopii} $\textbf{hTop}$, w której obiektami są przestrzenie topologiczne, ale morfizmami są klasy abstrakcji homotopijnych funkcji (złożenie można określić przez reprezentantów, a morfizmem identycznościowym jest klasa abstrakcji funkcji identycznościowej). Tutaj też morfizmami nie są funkcje, lecz klasy abstrakcji funkcji.
\end{exmp}

\begin{exmp}
  Każdą kolekcję (w tym każdy zbiór) można traktować jak kategorię - obiektami są elementy kolekcji i wprowadzamy wyłącznie morfizmy identycznościowe. Często się rysuje taką kategorię jako:
  $$\bullet~\bullet~\bullet~\dots $$
  Obiekty są symbolizowane przez kropki i nie ma żadnych morfizmów poza identycznościowymi (pominięte na rysunku). Taką kategorię nazywamy \emph{dyskretną}. Oczywiście jest lokalnie mała (a jeśli kolekcja obiektów jest zbiorem, to jest także mała).
\end{exmp}

\begin{exmp}
  \label{exmp:dwa}
  Kategorią \textbf{2} nazywamy kategorię, która ma dwa obiekty i jeden morfizm nieidentycznościowy. Rysujemy ją jako:
  $$\bullet \rightarrow \bullet$$
  (ponownie pominęliśmy na obrazku morfizmy identycznościowe).
  %
  Jest mała, a więc także lokalnie mała.
\end{exmp}

\begin{exc}[Kategoria $\SetStar$]
  \emph{Zbiorem z wyróżnionym punktem} nazywamy parę $(X, x_0)$, gdzie $x_0\in X$. Morfizmem $\morph f{(X, x_0)}{(Y, y_0)}$ nazywamy funkcję $\morph fXY$ taką, że $f(x_0)=y_0$. Przekonaj się, że jest to kategoria. (Czym jest złożenie morfizmów? Czy jest łączne? istnieją morfizmy identycznościowe?)
\end{exc}

\subsection{Zbiór częściowo uporządkowany jako kategoria}
\index{Zbiór częściowo uporządkowany}
\begin{idea}
  Podobnie jak ze zbioru można zrobić w oczywisty sposób kategorię dyskretną (dodając identyczności), tak każdy zbiór częściowo uporządkowany można zamienić w pewną kategorię (podobnie jak w przykładzie \ref{exmp:natur} o liczbach naturalnych). Takie kategorie zarówno stanowią bogate źródło kontrprzykładów (jak i przydają się do zdefiniowania presnopów i ich źdźbeł).
\end{idea}

\begin{exmp}
  \label{exmp:dzielenie}
  {\newcommand{\divs}[2]{\iota_{#1}^{#2}}
  Rozważmy zbiór dodatnich liczb całkowitych $\mathbb Z^+=\{1, 2, \dots\}$. Definiujemy zbiory:
  $$
  \Mor km =
  \begin{cases}
    \{ \divs km \} \text{ jeśli $k$ jest dzielnikiem $m$,}\\
    \varnothing \text{ w przeciwnym przypadku.}
  \end{cases}
  $$
  gdzie przyjęliśmy zapis $\divs km=(k, m)$.
  Definiujemy złożenie jako $\divs mn  \circ  \divs km = \divs kn$.
  Sprawdźmy czy wszystkie własności są spełnione:
  \begin{itemize}
    \item weźmy dowolny element $\Mor km$ (czyli $\divs km$) oraz dowolny element $\Mor mn$ (czyli $\divs mn$). Chcemy pokazać, że złożenie $\divs mn \circ \divs km = \divs kn$ jest dobrze określone, to znaczy, że $\divs kn \in \Mor kn$. Tłumaczenie: jeśli $k$ dzieli $m$ oraz $m$ dzieli $n$, to $k$ dzieli $n$.
    \item $\id n = \divs nn$ jest dobrze określone bo $n$ dzieli $n$. Chcemy pokazać, że mnożenie przez identyczność niczego nie zmienia, to znaczy:
      $$\divs km \circ \divs kk = \divs km = \divs mm \circ \divs km$$
      Co jest oczywiste z naszej definicji złożenia.
    \item łączność jest podobnie trywialna:
      $$(\divs mn \circ \divs lm) \circ  \divs kl  = \divs ln \circ \divs kl = \divs kn = \divs mn \circ \divs km = \divs mn \circ (\divs lm \circ  \divs kl)$$
  \end{itemize}
  }
\end{exmp}

\begin{exc}
  Przypomnijmy, że zbiór częściowo uporządkowany $(S, \le)$ ma następujące własności dla dowolnych $a, b, c \in S$:
  \begin{itemize}
    \item $a\le a$,
    \item jeśli $a\le b$ oraz $b\le a$, to $b=a$,
    \item jeśli $a\le b$ oraz $b\le c$, to $a\le c$,
  \end{itemize}
  %
  Posiłkując się powyższym przykładem \ref{exmp:dzielenie}, pokaż, że \emph{każdy zbiór częściowo uporządkowany jest kategorią} jeśli wprowadzimy strzałki $\morpharr{} ab$ dla $a\le b$.
\end{exc}

\begin{exmp}
    Kategoria \textbf{2} z przykładu \ref{exmp:dwa} jest szczególnym przypadkiem zbioru częściowo uporządkowanego.
\end{exmp}

\subsection{Diagramy przemienne}
\begin{defn}[Intuicyjna definicja diagramu]
  Podstawowym narzędziem teorii kategorii są \emph{diagramy}. Będąc niezbyt precyzyjnym, możemy powiedzieć, że są to multigrafy, których wierzchołkami są obiekty, a krawędziami morfizmy.
  Mówimy, że diagram jest \emph{przemienny} jeśli mając dowolne dwa wierzchołki $A$ i $B$ oraz dwie dowolne ścieżki $A\xrightarrow{f_1} X_1 \xrightarrow{f_2}\dots\xrightarrow{f_n} B$ i~$A\xrightarrow{g_1} Y_1 \xrightarrow{g_2}\dots\xrightarrow{g_m} B$, zachodzi równość $f_n\circ \dots \circ f_1 = g_m\circ \dots \circ g_1$. (Gdy ustalimy początek i koniec, nieważne jaką ścieżką pójdziemy, a wynikowy morfizm będzie taki sam.)
\end{defn}

\begin{exmp}
  Ten diagram jest przemienny wtedy i tylko wtedy gdy $g\circ f = i\circ h$.
  \begin{center}
    \begin{tikzcd}
      A \arrow[d, "h"] \arrow[r, "f"] &  B \arrow[d, "g"]\\
      C \arrow[r, "i"] & D
    \end{tikzcd}
  \end{center}
\end{exmp}

\begin{exc}
  Jaki aksjomat kategorii wyraża przemienność tego diagramu?
  \begin{center}
  \begin{tikzcd}
    A \ar[r, "f"] \ar[rr, bend left=50] & B \ar[r, "g"] \ar[rr, bend right=50] & C \ar[r, "h"] & D
  \end{tikzcd}
  \end{center}
\end{exc}

\begin{nott}
  Będziemy korzystać z przerywanej strzałki \begin{tikzcd} \bullet\arrow[r, dotted] & \bullet\end{tikzcd} postulując istnienie danego morfizmu.
\end{nott}

\begin{exmp} Diagram przemienny
  \begin{center}
  \begin{tikzcd}
    A \arrow[d, "f"] \arrow[r, dotted] &  C \\
    B \arrow[ru, "g"]
  \end{tikzcd}
  \end{center}
  postuluje istnienie strzałki $g\circ f$.
\end{exmp}

\begin{nott}
  Mówiąc, że \emph{istnieje dokładnie jedna} strzałka, która sprawia, że diagram jest przemienny, będziemy używać wykrzyknika \begin{tikzcd} \bullet\arrow["!", r, dotted] & \bullet\end{tikzcd}.
\end{nott}

\begin{exc}[$\Set_*$ dla bystrzaków]
  \label{exc:setstardlabystrzakow}
  {
  \newcommand{\bulletset}{\{\bullet\}}
  Aby nabrać więcej praktyki z diagramami przemiennymi, obejrzymy kategorię zbiorów z wyróżnionym punktem raz jeszcze, ale z nieco innego punktu widzenia.

  Wybierzmy dowolny jednopunktowy zbiór i oznaczmy go przez $\bulletset$. Jeśli $X$ jest niepustym zbiorem, to funkcja
  %\footnote{Raz jeszcze przypomnienie - przez funkcję rozumiemy trójkę (dziedzina, zbiór par (argument, wartość), przeciwdziedzina).}
  $\morph x\bulletset X$ wyznacza nam parę $(X, x_0)$, gdzie $x_0=x(\bullet)$. Innymi słowy obiektami są \emph{funkcje} ze zbioru $\bulletset$ w dowolne zbiory.

  Czym może być morfizm między funkcjami? Rozważmy $\morph x\bulletset X$ oraz $\morph y\bulletset Y$. Morfizmem $\morpharr fxy$ nazywamy \emph{diagram przemienny}:
  %
  \begin{center}
  \begin{tikzcd}
    &\bulletset \ar[ld, "x", swap] \ar[rd, "y"]\\
    X \ar[rr, "f"]& & Y
  \end{tikzcd}
  \end{center}
  %
  Jego przemienność oznacza $f(x_0) = (f\circ x)(\bullet) = y(\bullet) = y_0$! Alternatywnie, możemy myśleć o tym jak o funkcji $\morph fXY$ takiej, że $f(x_0)=y_0$.

  Zastanówmy się czym jest złożenie diagramów. Mając diagramy
  %
  \begin{center}
  \begin{tikzcd}
                   &\bulletset \ar[ld, "x", swap] \ar[rd, "y"] &   &               & \bulletset \ar[ld, "y", swap] \ar[rd, "z"]\\
    X \ar[rr, "f"] &                                           & Y & Y \ar[rr, "g"]&                                            & Z
  \end{tikzcd}
  \end{center}
  %
  określamy ich złożenie jako sklejenie wzdłuż wspólnej krawędzi:
  %
  \begin{center}
    \begin{tikzcd}
    & \bulletset \ar[ld, "x", swap] \ar[d, "y"] \ar[rd, "z"] & & & \bulletset \ar[ld, "x", swap] \ar[rd, "z"]\\
    X \ar[r, "f"] & Y \ar[r, "g"] & Z & X \ar[rr, "g\circ f"]& & Z
    \end{tikzcd}
  \end{center}
  %
  W ramach ćwiczenia:
  \begin{enumerate}
    \item Upewnij się, że $(g\circ f)(x_0) = z_0$.
    \item Narysuj diagram przedstawiający morfizm identycznościowy $\id x$. Czy identyczności ,,dobrze'' się składają?
    \item Pokaż, że składanie diagramów jest łączne.
  \end{enumerate}
  }
\end{exc}

% \begin{exc}[Kategoria strzałek]
%   % Kategoria strzałek
%   {
%   \newcommand{\ArrC}{\text{Arr}(\C)}
%   \newcommand{\MorC}{\text{Mor}(\C)}
%   Rozpatrzmy kategorię $\C$ i stwórzmy kolekcję wszystkich jej morfizmów $\MorC$. Stworzymy teraz nową kategorię $\ArrC$:
%   \begin{itemize}
%     \item $\Ob \ArrC = \MorC$ (czyli obiektami tej kategorii są strzałki $\morpharr fAB$),
%     \item morfizmem między strzałkami $\morpharr fAB$ oraz $\morpharr g {A'}{B'}$ nazywamy diagram przemienny:
%       \begin{center}
%       \begin{tikzcd}
%         A \ar[r, "f"] \ar[d, "a"] & B \ar[d, "b"]\\
%         A' \ar[r, "g"] & B'
%       \end{tikzcd}
%       \end{center}
%     \item mając dwa diagramy:
%       \begin{center}
%       \begin{tikzcd}
%         A \ar[r, "f"] \ar[d, "a"] & B \ar[d, "b"] & B  \ar[d, "b"] \ar[r, "h"] & \ar[d, "c"] C\\
%         A' \ar[r, "g"]            & B'            & B' \ar[r, "i"]             & C'
%       \end{tikzcd}
%       \end{center}
%       określamy ich złożenie jako sklejenie:
%       \begin{center}
%       \begin{tikzcd}
%         A \ar[r, "f"] \ar[d, "a"] & B  \ar[d, "b"] \ar[r, "h"] & \ar[d, "c"] C & A \ar[r, "h\circ f"] \ar[d, "a"] & C \ar[d, "c"]\\
%         A' \ar[r, "g"]            & B' \ar[r, "i"]             & C'            & A' \ar[r, "i\circ g"]            & C'
%       \end{tikzcd}
%       \end{center}
%
%   \end{itemize}
%   (Można myśleć o morfizmach $\Set_*$ jako o funkcjach zamieniających wyróżniony punkt na inny wyróżniony punkt - podobnie tutaj można myśleć o morfizmach jak o \emph{parach funkcji}.)
%
%   Pokaż, że jest to kategoria, czyli, że:
%   \begin{enumerate}
%     \item składanie morfizmów jest dobrze określone (dlaczego diagram przedstawiający złożenie jest przemienny?),
%     \item jest też łączne,
%     \item istnieją morfizmy identycznościowe, które się należycie składają.
%   \end{enumerate}
%   }
% \end{exc}
