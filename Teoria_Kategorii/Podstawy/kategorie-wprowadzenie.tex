\documentclass[12pt, notitlepage]{article}

\usepackage{polski}
\usepackage[utf8]{inputenc}
\usepackage{amsmath}
\usepackage{amssymb}
\usepackage{fancyhdr}
% \usepackage[a4paper,bindingoffset=0.2in,left=0.6in,right=0.6in,top=1in,bottom=1in,footskip=.1in]{geometry}
\usepackage{color}
\usepackage{xcolor}
\usepackage{graphicx}
\usepackage{subcaption}
\usepackage{hyperref}
\usepackage{rotating}
\usepackage{amsthm}
\usepackage{tikz-cd}

\swapnumbers
\theoremstyle{definition}

\newtheorem{thm}{Twierdzenie}
\newtheorem{lem}[thm]{Lemat}
\newtheorem{fct}[thm]{Fakt}
\newtheorem{cor}[thm]{Wniosek}
\newtheorem{idea}[thm]{Pomysł}
\newtheorem{defn}[thm]{Definicja}
\newtheorem{exc}[thm]{Ćwiczenie}
\newtheorem{prof}[thm]{Dowód}
\newtheorem{nott}[thm]{Notacja}
\newtheorem{noth}[thm]{}
\newtheorem{remk}[thm]{Uwaga}
\newtheorem{exmp}[thm]{Przykład}
\newtheorem{joke}[thm]{Żart}


\newcommand{\todo}{\textbf{~TODO!~}}

\newcommand{\Ob}[1]{\textrm{Ob\,}{#1}}
\newcommand{\bareMor}{\textrm{Hom}}
\newcommand{\Mor}[2]{\bareMor(#1, #2)}
\newcommand{\morph}[3]{#1\colon #2\to #3}
\newcommand{\morpharr}[3]{#2 \xrightarrow{#1} #3}
\newcommand{\nattran}[3]{#1\colon #2\Rightarrow #3}
\newcommand{\isomorph}[3]{#1: #2 \xrightarrow{~\simeq~} #3}
\newcommand{\id}[1]{\mathrm{id}_{#1}}
\newcommand{\op}[1]{{#1}^{op}}
\newcommand{\starred}{$\star$\,}

\newcommand{\actsas}[3]{#1\colon #2\mapsto #3}

\newcommand{\Set}{\textbf{Set}}
\newcommand{\SetStar}{\textbf{Set}_*}
\newcommand{\Top}{\textbf{Top}}
\newcommand{\FinSet}{\textbf{FinSet}}
\newcommand{\Ab}{\textbf{Ab}}
\newcommand{\Grp}{\textbf{Grp}}
\newcommand{\Mod}[1]{#1\textbf{-Mod}}
\newcommand{\Vect}[1]{#1\textbf{-Vect}}
\newcommand{\FinVect}[1]{#1\textbf{-FinVect}}

\newcommand{\C}{\mathcal C}
\newcommand{\D}{\mathcal D}

\title{Teoria kategorii\\
\large dla początkujących}
\author{Paweł Czyż}

\begin{document}
  \maketitle
  \begin{abstract}
    Pewne pojęcia, jak zbiór czy grupa, należą do kanonu matematyki -- matematycy korzystają z nich na tyle często, że każdy kiedyś musiał się z nimi spotkać.

    Tak samo często matematycy korzystają z pojęć teorii kategorii, choć nie wszyscy o tym wiedzą. W tym tekście próbuję przybliżyć podstawy teorii kategorii, tak by zarówno zwiększyć popularność ,,kategoryjnego'' myślenia jak i zachęcić do dalszych studiów tej dziedziny.

    Zakładam trochę obycia z podstawową teorią mnogości i styczność teorią grup, algebrą liniową lub topologią ogólną. Materiał wychodzący poza te oznaczenia oznaczyłem gwiazdką \starred.
  \end{abstract}

 \section{Czym są kategorie?}
\begin{remk}
  Zbiory intuicyjnie rozumiane są jako pewne kolekcje, czyli woreczki, które mogą zawierać różne przedmioty. Taka intuicja czasami jednak prowadzi na manowce -- nie istnieje zbiór wszystkich zbiorów. Tymczasem zwrot ,,kolekcja wszystkich zbiorów'' jest intuicyjnie zrozumiały -- mimo, że nie może być zbiorem!

  Bardzo niefrasobliwie, nie będziemy przejmować się aksjomatyką ,,kolekcji''. Nasze notatki nie będą więc ścisłe w sensie teoriomnogościowym. Nie będziemy się tym przejmować -- szczegółów można douczyć się później, na razie wystarczy zrozumieć ideę.

  Czytelnik bardzo zainteresowany teorią mnogości, może zainteresować się \emph{uniwersami Grothendiecka} lub teoriami \emph{klas}, jak aksjomatyka NBG. Można też przyjąć, że język teorii kategorii jest ,,prawdziwy'' i w nim zbudować teorię zbiorów \cite{rethinking_set_theory}.
\end{remk}

\begin{idea}
  Zbiór to kolekcja elementów. \emph{Kategoria} to kolekcja obiektów i~strzałek, które łączą różne obiekty -- na przykład kolekcja wszystkich zbiorów połączona funkcjami, czy kolekcja wszystkich grup połączona homomorfizmami.

  Innymi słowy, zamiast odpowiadać na teoriomnogościowe pytanie o pojedynczy przedmiot: ,,jakie elementy ma ten zbiór?'' kładziemy nacisk na całościowe spojrzenie: ,,w jaki sposób ten obiekt oddziałuje z innymi?''.
\end{idea}

\begin{defn}
  \index{Kategoria}
  Będziemy nazywać $\C$ \emph{kategorią} jeśli:
    \begin{enumerate}
      \item Mamy pewną kolekcję \emph{obiektów} $A, B, C, \dots$ Będziemy oznaczać tę kolekcję przez $\Ob \C$,
      \item Mamy pewną kolekcję \emph{strzałek} $f, g, h\dots$ Strzałki będziemy nazywać też \emph{morfizmami},
      \item Każda strzałka $f$ łączy dokładnie dwa określone obiekty -- \emph{dziedzinę} z~\emph{przeciwdziedziną}. Ten fakt oznaczamy przez $\morph fAB$ lub $\morpharr fAB$,
      \item Strzałki można \emph{składać} w następujący sposób:
        \begin{enumerate}
          \item Jeśli $\morpharr fAB$ oraz $\morpharr gBC$, to istnieje strzałka $\morpharr {g\circ f}AC$. Będziemy czasami pisać $gf$ zamiast $g\circ f$,
          \item Jeśli $\morpharr fAB$, $\morpharr gBC$ oraz $\morpharr h CD$, to zachodzi $(h\circ g)\circ f = h\circ (g\circ f)$ (złożenie jest \emph{łączne}),
          \item Dla każdego obiektu $A$ istnieje strzałka $\morph{\id A}AA$ taka, że dla każdej $\morph fAB$ zachodzi $\id B \circ f = f = f \circ \id A$ (istnieją \emph{morfizmy identycznościowe}).
        \end{enumerate}
    \end{enumerate}
\end{defn}

\begin{exmp}
  \label{exmp:set}
  Kategorią $\Set$ nazywamy kolekcję wszystkich zbiorów, w której strzałka $\morpharr fAB$ to trójka $(A, f, B)$, gdzie $f$ jest funkcją ze zbioru $A$ w zbiór $B$. Złożenie strzałek określamy przez złożenie funkcji, a strzałki identycznościowe otrzymujemy z funkcji identycznościowych.

  Pojawia się pytanie -- dlaczego definiujemy strzałkę jako \emph{trójkę} zamiast po prostu funkcję? Teoriomnogościowo funkcje $\morph f{\mathbb R}{\mathbb R}$ oraz $\morph {f'}{\mathbb R}{\mathbb Z}$ zadane wzorem $x\mapsto 1$ są tym samym -- zbiorem $\mathbb R\times \{1\}$. W teorii kategorii jednak chcemy rozróżniać $f$ i $f'$, bo łączą różne zbiory. Dlatego ręcznie dodajemy informację o dziedzinie i~przeciwdziedzinie.

  Jako, że ręczne dodanie informacji jest proste, to popularnie (i nie do końca ściśle) mówi się, że $\Set$ to kategoria w której obiektami są zbiory, a~morfizmami są funkcje.
\end{exmp}

\begin{exmp}[Inne ,,standardowe'' kategorie]
  \label{exmp:standard}
  Podobnie jak w powyższym przykładzie, mamy kategorie (uwaga -- tak teoriomnogościowo to morfizmami są trójki, a nie funkcje):
  \begin{enumerate}
    % \item $\FinSet$ - obiektami są zbiory skończone, a morfizmami - funkcje,
    \item $\Grp$ -- obiektami są grupy, a morfizmami homomomorfizmy grup (składanie homomomorfizmów jest łączne, identycznościami są homomomorfizmy identycznościowe),
    \item $\Top$ -- obiektami są przestrzenie topologiczne, a morfizmami są funkcje ciągłe,
    % \item $\Mod R$ - obiektami są lewostronne moduły nad pierścieniem $R$, a morfizmami funkcje $R$-liniowe (analogicznie prawe moduły tworzą kategorię),
    \item $\Vect k$ -- obiektami są przestrzenie wektorowe nad ciałem $k$, a morfizmami są funkcje liniowe,
    % \item $\Ab$ - obiektami są grupy abelowe, a morfizmami homomomorfizmy grup (to też szczególny przypadek kategorii modułów - $R=\mathbb Z$),
    \item $\FinVect k$ -- obiektami są przestrzenie wektorowe skończonego wymiaru nad ciałem $k$, a morfizmami są funkcje liniowe.
  \end{enumerate}
  %
  Podobnych przykładów można wymieniać bez liku (monoidy, pierścienie, moduły nad ustalonym pierścieniem, rozmaitości różniczkowe\dots) -- często obiekty to po prostu zbiory wyposażone w dodatkową strukturę, a morfizmami są funkcje zachowujące tę strukturę. Należy jednak pamiętać, że \emph{kategorie obejmują także inne przykłady}.
\end{exmp}

\begin{exmp}
  Rozważmy kategorię, której obiektami są grupy, a morfizmami funkcje, niezależnie czy są homomorfizmami czy nie. To też jest kategoria, choć niezbyt użyteczna -- we wszystkich powyższych, popularnych kategoriach morfizmy w pewien sposób zachowują zadaną stukturę (jak działanie grupowe czy topologia).
\end{exmp}

\begin{exmp}
  Każdą kolekcję (w tym każdy zbiór) można traktować jak kategorię - obiektami są elementy kolekcji oraz wprowadzamy wyłącznie morfizmy identycznościowe. Często się rysuje taką kategorię jako:
  $$\bullet~\bullet~\bullet~\dots $$
  Obiekty są symbolizowane przez kropki i nie ma żadnych morfizmów poza identycznościowymi (pominiętymi na rysunku). Taką kategorię nazywamy \emph{dyskretną}.
\end{exmp}

\begin{exmp}
  \label{exmp:dwa}
  Kategorią \textbf{2} nazywamy kategorię, która ma dwa obiekty i jeden morfizm nieidentycznościowy. Rysujemy ją jako:
  $$\bullet \rightarrow \bullet$$
  Ponownie pominęliśmy na obrazku morfizmy identycznościowe.
\end{exmp}

\begin{nott}
  $\Mor AB$ to kolekcja wszystkich strzałek $\morpharr fAB$. W literaturze popularne są także oznaczenia\footnote{Dlaczego Hom, a nie Mor? Zapewne dlatego, że np. w teorii grup czy pierścieni, na strzałki mówi się \emph{homomorfizmy}.} $\bareMor_\C(A, B)$ lub $\mathrm{Mor}(A, B)$.
\end{nott}

\begin{defn}
  Kategorię nazywamy \emph{małą} jeśli kolekcje strzałek i obiektów są zbiorami.
  Nazywamy ją \emph{lokalnie małą} jeśli wszystkie kolekcje $\Mor AB$ są zbiorami.
\end{defn}

\begin{exc}
  Zastanów się:
  \begin{enumerate}
    \item Czy możliwe jest aby kolekcja strzałek nie była zbiorem, a kolekcja obiektów już tak?
    \item Czy możliwe jest aby kolekcja obiektów nie była zbiorem, a kolekcja strzałek już tak? Czyli -- czy obiektów może być ,,więcej'' niż strzałek?
    \item Dlaczego każda kategoria mała jest też lokalnie mała, ale nie na odwrót?
  \end{enumerate}
\end{exc}

% \begin{exmp}
%   Kategorie z przykładu \ref{exmp:standard} są lokalnie małe, lecz nie są małe.
% \end{exmp}

% \begin{remk}[Uwaga techniczna, można pominąć]
%   \label{remk:rozlaczne}
%   Warto zauważyć, że $\Mor AB$ oraz $\Mor{A'}{B'}$ są albo identyczne (gdy $A=A'$ i $B=B'$) albo rozłączne (w przeciwnym przypadku).
%   To wymaganie jest czysto techniczne (każda strzałka musi łączyć dokładnie dwa obiekty) - mając zbiory $\bareMor'(A, B)$, które nie są parami rozłączne możemy zastąpić je przez $\Mor AB =\{A\}\times \bareMor'(A, B) \times \{B\}$, to znaczy zamiast rozpatrywać $\morph fAB$ wystarczy rozpatrywać trójki uporządkowane $(A, f, B)$, tak samo jak w przykładzie \ref{exmp:set} o kategorii $\Set$.
% \end{remk}

% \begin{exmp}[\starred Dla topologów]
%   Można też definiować tak zwaną \emph{naiwną kategorię homotopii} $\textbf{hTop}$, w której obiektami są przestrzenie topologiczne, ale morfizmami są klasy abstrakcji homotopijnych funkcji (złożenie można określić przez reprezentantów, a morfizmem identycznościowym jest klasa abstrakcji funkcji identycznościowej). Tutaj też morfizmami nie są funkcje, lecz klasy abstrakcji funkcji.
% \end{exmp}

\begin{exc}[Kategoria $\SetStar$]
  \emph{Zbiorem z wyróżnionym punktem} nazywamy parę $(X, x_0)$, gdzie $x_0\in X$. Morfizmem $\morph f{(X, x_0)}{(Y, y_0)}$ nazywamy funkcję $\morph fXY$ taką, że $f(x_0)=y_0$. Przekonaj się, że jest to kategoria. (Czym jest złożenie morfizmów? Czy jest łączne? Czym są morfizmy identycznościowe?)
\end{exc}

\subsection{Zbiór częściowo uporządkowany jako kategoria}
\index{Zbiór częściowo uporządkowany}
\begin{idea}
  Każdy zbiór w naturalny sposób jest kategorią dyskretną -- morfizmami są wyłącznie identyczności. Nieco ciekawszą strukturę, z większą ilością strzałek, możemy uzyskać ze zbioru częściowo uporządkowanego.

  Takie kategorie zarówno stanowią bogate źródło kontrprzykładów jak i~przydają się do zdefiniowania bardziej zaawansowanych konstrukcji, jak granica odwrotna czy źdźbło presnopa.
\end{idea}

\begin{exmp}
  \label{exmp:dzielenie}
  {\newcommand{\divs}[2]{\iota_{#1}^{#2}}
  Rozważmy zbiór dodatnich liczb całkowitych $\mathbb Z^+=\{1, 2, \dots\}$. Definiujemy zbiory:
  $$
  \Mor km =
  \begin{cases}
    \{ \divs km \} &\text{ jeśli $k$ jest dzielnikiem $m$}\\
    \varnothing    &\text{ w przeciwnym przypadku}
  \end{cases}
  $$
  gdzie przyjęliśmy zapis $\divs km=(k, m)$.
  Definiujemy złożenie jako $\divs mn  \circ  \divs km = \divs kn$.
  Sprawdźmy czy wszystkie własności są spełnione:
  \begin{enumerate}
    \item \emph{Złożenie jest określone.} Weźmy dowolny element $\Mor km$ (dużego wyboru nie mamy -- $\divs km$) oraz dowolny element $\Mor mn$ (czyli $\divs mn$). Chcemy pokazać, że złożenie $\divs mn \circ \divs km = \divs kn$ jest dobrze określone, to znaczy, że $\divs kn \in \Mor kn$. Czytamy to jako: ,,jeśli $k$ dzieli $m$ oraz $m$ dzieli $n$, to $k$ dzieli $n$''.
    \item \emph{Istnieją funkcje identycznościowe.} $\id n = \divs nn$ jest dobrze określone bo $n$ dzieli $n$. Chcemy pokazać, że mnożenie przez identyczność niczego nie zmienia, to znaczy:
      $$\divs km \circ \divs kk = \divs km = \divs mm \circ \divs km$$
      Co jest oczywiste z naszej definicji złożenia.
    \item \emph{Złożenie jest łączne.} $$(\divs mn \circ \divs lm) \circ  \divs kl  = \divs ln \circ \divs kl = \divs kn = \divs mn \circ \divs km = \divs mn \circ (\divs lm \circ  \divs kl)$$
  \end{enumerate}
  }
\end{exmp}

\begin{exc}
  Przypomnijmy, że zbiór częściowo uporządkowany $(S, \le)$ ma następujące własności dla dowolnych $a, b, c \in S$:
  \begin{enumerate}
    \item $a\le a$,
    \item Jeśli $a\le b$ oraz $b\le a$, to $b=a$,
    \item Jeśli $a\le b$ oraz $b\le c$, to $a\le c$.
  \end{enumerate}
  %
  Posiłkując się przykładem \ref{exmp:dzielenie} pokaż, że \emph{każdy zbiór częściowo uporządkowany jest kategorią} jeśli wprowadzimy strzałki $\morpharr{} ab$ dla $a\le b$.
\end{exc}

\begin{exmp}
    Kategoria \textbf{2} z przykładu \ref{exmp:dwa} jest szczególnym przypadkiem zbioru częściowo uporządkowanego.
\end{exmp}

\begin{exmp}
  \label{exmp:natur}
  {\newcommand{\naturalnum}{\mathbb N}
  Liczby naturalne ze standardowym porządkiem są kategorią:
  \begin{center}
  \begin{tikzcd}
    1 \ar[r] \ar[rr, bend left] \ar[rrr, bend left=50] & 2 \ar[r] \ar[rr, bend left] & 3 \ar[r] & \dots
  \end{tikzcd}
  \end{center}
  Tradycyjnie na obrazku brakuje morfizmów identycznościowych (które wyglądają jak pętelki).
  }
\end{exmp}

\subsection{Diagramy przemienne}
\begin{defn}[Intuicyjna definicja diagramu]
  \emph{Diagram} to multigraf, którego wierzchołkami są obiekty, a krawędziami morfizmy.

  Mówimy, że diagram jest \emph{przemienny} jeśli mając dowolne dwa wierzchołki $A$ i $B$ oraz dwie dowolne ścieżki (niekoniecznie tej samej długości):

  \begin{center}
    \begin{tikzcd}
      & X_1 \ar[r, "f_2"] & X_2 \ar[r, "f_3"] & \dots \ar[r, "f_n"]& X_n \ar[dr, "f_{n+1}"]\\
      A \arrow[dr, "g_1", swap] \arrow[ru, "f_1"] & & &  & & B \\
      & Y_1 \ar[r, "g_2"] & Y_2 \ar[r, "g_3"]& \dots \ar[r, "g_m"]& Y_m\ar[ur, "g_{m+1}", swap]
    \end{tikzcd}
  \end{center}
  zachodzi równość $f_{n+1}\circ \dots \circ f_1 = g_{m+1}\circ \dots \circ g_1$.

  Innymi słowy -- gdy ustalimy początek i koniec, to nieważne jaką ścieżką pójdziemy, a wypadkowy morfizm będzie taki sam.
\end{defn}

\begin{exmp}
  Ten diagram jest przemienny wtedy i tylko wtedy gdy $g\circ f = i\circ h$.
  \begin{center}
    \begin{tikzcd}
      A \arrow[d, "h"] \arrow[r, "f"] &  B \arrow[d, "g"]\\
      C \arrow[r, "i"] & D
    \end{tikzcd}
  \end{center}
\end{exmp}

\begin{exc}
  Jaki aksjomat kategorii wyraża przemienność tego diagramu?
  \begin{center}
  \begin{tikzcd}
    A \ar[r, "f"] \ar[rr, bend left=50] & B \ar[r, "g"] \ar[rr, bend right=50] & C \ar[r, "h"] & D
  \end{tikzcd}
  \end{center}
\end{exc}

\begin{nott}
  Będziemy korzystać z przerywanej strzałki \begin{tikzcd} \bullet\arrow[r, dotted] & \bullet\end{tikzcd} postulując istnienie danego morfizmu.
\end{nott}

\begin{exmp} Diagram przemienny
  \begin{center}
  \begin{tikzcd}
    A \arrow[d, "f", swap] \arrow[r, dotted] &  C \\
    B \arrow[ru, "g", swap]
  \end{tikzcd}
  \end{center}
  postuluje istnienie strzałki $g\circ f$.
\end{exmp}

\begin{nott}
  Mówiąc, że \emph{istnieje dokładnie jedna} strzałka, która sprawia, że diagram jest przemienny, będziemy używać wykrzyknika \begin{tikzcd} \bullet\arrow["!", r, dotted] & \bullet\end{tikzcd}.
\end{nott}

\begin{exc}[$\Set_*$ dla bystrzaków]
  \label{exc:setstardlabystrzakow}
  {
  \newcommand{\bulletset}{\{\bullet\}}
  Aby nabrać więcej praktyki z diagramami przemiennymi, obejrzymy kategorię zbiorów z wyróżnionym punktem raz jeszcze, ale z nieco innego punktu widzenia.

  Wybierzmy dowolny jednopunktowy zbiór i oznaczmy go przez $\bulletset$. Jeśli $X$ jest niepustym zbiorem, to funkcja
  %\footnote{Raz jeszcze przypomnienie - przez funkcję rozumiemy trójkę (dziedzina, zbiór par (argument, wartość), przeciwdziedzina).}
  $\morph x\bulletset X$ wyznacza nam parę $(X, x_0)$, gdzie $x_0=x(\bullet)$. Innymi słowy obiektami są \emph{funkcje} ze zbioru $\bulletset$ w dowolne zbiory.

  Czym może być morfizm między funkcjami? Rozważmy $\morph x\bulletset X$ oraz $\morph y\bulletset Y$. Morfizmem $\morpharr fxy$ nazywamy \emph{diagram przemienny}:
  %
  \begin{center}
  \begin{tikzcd}
    &\bulletset \ar[ld, "x", swap] \ar[rd, "y"]\\
    X \ar[rr, "f"]& & Y
  \end{tikzcd}
  \end{center}
  %
  Jego przemienność oznacza $f(x_0) = (f\circ x)(\bullet) = y(\bullet) = y_0$. Alternatywnie, możemy myśleć o tym diagramie jak o funkcji $\morph fXY$ takiej, że $f(x_0)=y_0$.

  Zastanówmy się czym jest złożenie diagramów. Mając diagramy
  %
  \begin{center}
  \begin{tikzcd}
                   &\bulletset \ar[ld, "x", swap] \ar[rd, "y"] &   &               & \bulletset \ar[ld, "y", swap] \ar[rd, "z"]\\
    X \ar[rr, "f"] &                                           & Y & Y \ar[rr, "g"]&                                            & Z
  \end{tikzcd}
  \end{center}
  %
  określamy ich złożenie jako sklejenie wzdłuż wspólnej krawędzi:
  %
  \begin{center}
    \begin{tikzcd}
    & \bulletset \ar[ld, "x", swap] \ar[d, "y"] \ar[rd, "z"] & & & \bulletset \ar[ld, "x", swap] \ar[rd, "z"]\\
    X \ar[r, "f"] & Y \ar[r, "g"] & Z & X \ar[rr, "g\circ f"]& & Z
    \end{tikzcd}
  \end{center}
  %
  W ramach ćwiczenia:
  \begin{enumerate}
    \item Upewnij się, że $(g\circ f)(x_0) = z_0$.
    \item Narysuj diagram przedstawiający morfizm identycznościowy $\id x$. Czy identyczności ,,dobrze'' się składają?
    \item Pokaż, że składanie diagramów jest łączne.
  \end{enumerate}
  }
\end{exc}

% \begin{exc}[Kategoria strzałek]
%   % Kategoria strzałek
%   {
%   \newcommand{\ArrC}{\text{Arr}(\C)}
%   \newcommand{\MorC}{\text{Mor}(\C)}
%   Rozpatrzmy kategorię $\C$ i stwórzmy kolekcję wszystkich jej morfizmów $\MorC$. Stworzymy teraz nową kategorię $\ArrC$:
%   \begin{itemize}
%     \item $\Ob \ArrC = \MorC$ (czyli obiektami tej kategorii są strzałki $\morpharr fAB$),
%     \item morfizmem między strzałkami $\morpharr fAB$ oraz $\morpharr g {A'}{B'}$ nazywamy diagram przemienny:
%       \begin{center}
%       \begin{tikzcd}
%         A \ar[r, "f"] \ar[d, "a"] & B \ar[d, "b"]\\
%         A' \ar[r, "g"] & B'
%       \end{tikzcd}
%       \end{center}
%     \item mając dwa diagramy:
%       \begin{center}
%       \begin{tikzcd}
%         A \ar[r, "f"] \ar[d, "a"] & B \ar[d, "b"] & B  \ar[d, "b"] \ar[r, "h"] & \ar[d, "c"] C\\
%         A' \ar[r, "g"]            & B'            & B' \ar[r, "i"]             & C'
%       \end{tikzcd}
%       \end{center}
%       określamy ich złożenie jako sklejenie:
%       \begin{center}
%       \begin{tikzcd}
%         A \ar[r, "f"] \ar[d, "a"] & B  \ar[d, "b"] \ar[r, "h"] & \ar[d, "c"] C & A \ar[r, "h\circ f"] \ar[d, "a"] & C \ar[d, "c"]\\
%         A' \ar[r, "g"]            & B' \ar[r, "i"]             & C'            & A' \ar[r, "i\circ g"]            & C'
%       \end{tikzcd}
%       \end{center}
%
%   \end{itemize}
%   (Można myśleć o morfizmach $\Set_*$ jako o funkcjach zamieniających wyróżniony punkt na inny wyróżniony punkt - podobnie tutaj można myśleć o morfizmach jak o \emph{parach funkcji}.)
%
%   Pokaż, że jest to kategoria, czyli, że:
%   \begin{enumerate}
%     \item składanie morfizmów jest dobrze określone (dlaczego diagram przedstawiający złożenie jest przemienny?),
%     \item jest też łączne,
%     \item istnieją morfizmy identycznościowe, które się należycie składają.
%   \end{enumerate}
%   }
% \end{exc}

 \section{Proste konstrukcje uniwersalne}
\subsection{Izomorfizmy}
\begin{idea}
  W teorii mnogości zbiory są równe gdy mają równe elementy. Tymczasem, teoria kategorii kładzie nacisk na obiekty i morfizmy między nimi - często będziemy uznawali za ,,równe'' obiekty, które po prostu ,,zachowują się tak samo''.
\end{idea}

\begin{exmp}
  Ile jest grup dwuelementowych? Nieskończenie wiele:
  \begin{itemize}
    \item $\{0, 1\}$ z~dodawaniem modulo 2,
    \item $\{0, 2\}$ z dodawaniem modulo 4,
    \item $\{-1, 1\}$ z mnożeniem,
    \item $\{\square, \blacksquare\}$ z operacją $\square \times \square = \blacksquare \times \blacksquare = \square$ i $\square \times \blacksquare = \blacksquare\times \square = \blacksquare$,
    \item ... (wiele, wiele ,,innych'', izomorficznych grup)
  \end{itemize}
  Tak naprawdę ,,inność'' bierze się wyłącznie z użycia innych elementów - mając twierdzenie o dowolnej z tych grup, bez problemu da się je przetłumaczyć na twierdzenie o dowolnej innej. Stąd właśnie pomysł na przymykanie oka na różnice między izomorficznymi obiektami - czym jest jednak izomorfizm poza teorią grup?
\end{exmp}

\begin{defn}
  \label{defn:izomorfizm}
  \index{Izomorfizm}
  Morfizm $\morph fAB$ nazywamy \emph{izomorfizmem} jeśli istnieje $\morph gBA$ takie, że $g\circ f = \id A$ oraz $f\circ g = \id B$.
\end{defn}

\begin{exmp}
  Izomorfizmami w $\Set$ są bijekcje. Tak samo w $\textbf{FinSet}$. W $\Grp$ są to bijektywne homomomorfizmy grup. W $\Vect k$ i $\FinVect k$ - bijektywne przekształcenia liniowe. W $\Top$ izomorfizmami są homeomorfizmy.
\end{exmp}

\begin{exc}
  Pokaż, że w zbiorze częściowo uporządkowanym jedynymi izomorfizmami są identyczności.
\end{exc}

\begin{exc}
  Pokaż, że $g$ występujące w definicji \ref{defn:izomorfizm} jest unikatowe. Motywuje to pojęcie \emph{morfizmu odwrotnego do $f$} i oznaczanego przez $f^{-1}$.
\end{exc}

\begin{exmp}
  Niech $G$ będzie grupą. Możemy stworzyć kategorię o jednym obiekcie $\bullet$, której morfizmami są elementy grupy $g, h,\dots$ traktowane jak strzałki $\morpharr g\bullet\bullet, \morpharr h\bullet\bullet, \dots$ - w tej kategorii każdy morfizm jest izomorfizmem.
\end{exmp}

\begin{exc}
  Pokaż, że:
  \begin{enumerate}
    \item $\id A$ jest izomorfizmem dla dowolnego obiektu $A$,
    \item jeśli $\morph fAB$ jest izomorfizmem, to $\morph {f^{-1}}BA$ też jest izomorfizmem,
    \item jeśli $\morph fAB$ oraz $\morph gBC$ są izomorfizmami, to $g\circ f$ też jest izomorfizmem.
  \end{enumerate}
\end{exc}

\begin{defn}
  Jeśli istnieje izomorfizm z $A$ do $B$, to będziemy mówili, że \emph{$A$ i~$B$ są izomorficzne} i pisali $A\simeq B$ lub rysowali na diagramie $A\xrightarrow{~\simeq~} B$.
\end{defn}

\begin{remk}
  Nietrudno zauważyć, że ,,bycie obiektem izomorficznym'' to relacja równoważności\footnote{Po prawdzie to klasy abstrakcji mogą być kolekcjami, a nie zbiorami, więc formalnie \emph{nie} jest to relacja równoważności...}. Dlatego w teorii kategorii często przymykamy oko na różnice między izomorficznymi obiektami - ,,zachowują się tak samo''.
\end{remk}

\begin{exc}[,,Zachowują się tak samo'']
  \label{exc:izomorficzne_obiekty}
  Dla ścisłości teoriomnogościowej, pracujemy w kategorii lokalnie małej (wszystkie $\Mor AB$ są zbiorami). Załóżmy, że mamy zadany izomorfizm $\morph fA{A'}$. Wykaż, że:
  \begin{enumerate}
    \item dla dowolnego $B$, mamy bijekcję $\morph{f^*}{\Mor {A'}B}{\Mor AB}$ zadaną wzorem $f^*(\psi) = \psi \circ f$ (co jest jej odwrotnością?),
    \item ponadto jeśli diagram:
      %
      \begin{center}
      \begin{tikzcd}
        & A' \arrow[ld, "\varphi" description, bend left=25] \arrow[d, "\psi"] \\
        B \arrow[r, "k", swap] & C
      \end{tikzcd}
      \end{center}
     jest przemienny, to:
     \begin{center}
     \begin{tikzcd}
         A \ar["f", r] \ar["\varphi \circ f", d, swap] \ar["\psi \circ f", rd, bend left=25, swap]  & A' \arrow[ld, "\varphi" description, bend left=25] \arrow[d, "\psi"] \\
         B \arrow[r, "k", swap] & C
       \end{tikzcd}
    \end{center}
    %
    też jest przemienny.
  \end{enumerate}
  %
  Innymi słowy mamy bijekcję między morfizmami zaczynającymi się w $A$ oraz zaczynającymi się w $A'$. Podobnie definiując $\morph{f_* }{\Mor BA}{\Mor B{A'}}$ przez $f_*(\psi) = f\circ \psi$ mamy bijekcję między morfizmami o końcu w $A$ oraz morfizmami o końcu w $A'$.
\end{exc}

\begin{remk}
  Całkiem naturalne jest by nasze konstrukcje nie rozróżniały między izomorficznymi obiektami. Dobrą intuicją może być tutaj topologia - definiujemy niezmienniki topologiczne (np. spójność, zwartość) tak, że są takie same dla homeomorficznych przestrzeni. Podobnie, jeśli $G\simeq G'$ oraz $H\simeq H'$ są grupami (lub przestrzeniami topologicznymi), to $G\times H \simeq G'\times H'$ - nie jest to przypadek i zazwyczaj widząc taką zależność należy się spodziewać konstrukcji kategoryjnej.
\end{remk}

\subsection{Obiekt początkowy i końcowy}
\begin{defn}
  Obiekt $P$ nazywamy \emph{początkowym} jeśli dla każdego obiektu $A$ \emph{istnieje dokładnie jeden} morfizm $P\to A$. Obiekt początkowy wygląda na diagramie jak \begin{tikzcd} P \ar[r, "!", dashed] & A \end{tikzcd}.
\end{defn}

\begin{exmp}
  W $\Set$ obiektem początkowym jest tylko zbiór pusty $\varnothing$. Z kolei w $\Grp$ obiektem początkowym jest każda grupa trywialna (z~jednym elementem) - czyli jest ich bardzo dużo. Wszystkie jednak są izomorficzne. Podobnie w kategoriach $\Vect k$ i $\FinVect k$.
\end{exmp}

\begin{exc}
	Znajdź kategorię, w której \emph{nie istnieje} obiekt początkowy.
\end{exc}

\begin{exc}
	Widzimy, że obiekty początkowe zazwyczaj nie są unikatowe w sensie teoriomnogościowym. Na szczęście są ,,prawie unikatowe'' to znaczy - ,,z dokładnością do izomorfizmu''. Wykaż, że:
	\begin{itemize}
		\item jeśli $P$ i $P'$ są obiektami początkowymi, to \emph{istnieje dokładnie jeden izomorfizm} $P\to P'$,
		\item jeśli $P$ jest obiektem początkowym i $A\simeq P$, to $A$ też jest obiektem początkowym.
	\end{itemize}
\end{exc}

\begin{remk}
	Zazwyczaj jeśli $A\simeq B$, to istnieje całkiem dużo izomorfizmów $A\to B$ (ile istnieje bijekcji między dwoma zbiorami pięcioelementowymi?!). Widzimy jednak, że mając dwa obiekty początkowe, izomorfizm między nimi jest \emph{unikatowy}.
	(Jeśli masz własny ulubiony obiekt początkowy i jakieś twierdzenie o nim, oraz ja mam swój ulubiony obiekt początkowy, to nie musimy się dogadać którego izomorfizmu użyć do przeniesienia rezultatu twierdzenia na mój obiekt.)
\end{remk}

\begin{defn}
	Odwróćmy teraz kierunek strzałki - obiekt $K$ nazywamy \emph{końcowym} jeśli dla każdego obiektu $A$ \emph{istnieje dokładnie jeden} morfizm $A\to K$. Zapisując obrazkiem: \begin{tikzcd} A \ar[r, "!", dashed] & K \end{tikzcd}
\end{defn}

\begin{exmp}
  W $\Set$ obiektem końcowym jest dowolny singleton $\{\bullet\}$. W $\Grp$ obiektem końcowym jest każda grupa trywialna. W $\Vect k$ i $\FinVect k$ obiektem końcowym jest każda trywialna przestrzeń wektorowa.
\end{exmp}

\begin{exc}
  Wykaż twierdzenie:
  \begin{enumerate}
    \item jeśli $K$ i $K'$ są obiektami końcowymi to istnieje dokładnie jeden izomorfizm $K\to K'$,
    \item jeśli $A\simeq K$, to $A$ też jest obiektem końcowym.
  \end{enumerate}
\end{exc}

\begin{defn}
  Jeśli obiekt $Z$ jest jednocześnie obiektem początkowym i końcowym, to nazywamy go obiektem \emph{zerowym}. Często jest też oznaczany przez 0.
\end{defn}

\begin{exmp}
  W $\Set$ \emph{nie ma} obiektu zerowego.
\end{exmp}

\begin{exmp}
  W $\Grp$ obiektem zerowym jest grupa trywialna. Podobnie w $\Ab$. Analogicznie w $\Vect k$, $\FinVect k$ i $\Mod R$ jest to przestrzeń trywialna.
\end{exmp}

\begin{exc}
  Jeśli istnieje obiekt zerowy, to wszystkie obiekty początkowe są zerowe. Podobnie wszystkie obiekty końcowe są wtedy zerowe.
\end{exc}

\subsection{Kategoria dualna}
\index{Kategoria dualna}
\begin{defn}
  \emph{Kategorią dualną} $\op \C$ nazywamy kategorię z tymi samymi obiektami i odwróconymi strzałkami. (Czyli $\Ob \C = \Ob {\op \C}$ oraz jeśli mamy $\morpharr fAB$ w $\C$, to definiujemy strzałkę $\morpharr fBA$ w $\op \C$.)
\end{defn}

\begin{exmp}
	Obiekt $P$ jest początkowy w $\C$ wtedy i tylko wtedy gdy jest końcowy w $\op \C$. Mówimy, że obiekty końcowe i początkowe są \emph{pojęciami dualnymi}.
\end{exmp}

\begin{defn}
	Rozważmy dowolną konstrukcję wyrażoną przy pomocy diagramów. Odwracając wszystkie strzałki otrzymujemy konstrukcję \emph{dualną}. Często dodaje się przedrostek ,,ko-'' do konstrukcji dualnej.
\end{defn}

\begin{joke}
	Obiekt początkowy zwany jest też ,,ńcowym''.
\end{joke}

\begin{exc}
	Uzasadnij dlaczego izomorfizm jest pojęciem dualnym sam do siebie. (Pokaż, że jeśli $f$ jest izomorfizmem w $\C$, to jest też w $\op \C$.)
\end{exc}

\subsection{Produkt}
\begin{defn}
	\emph{Produktem} obiektów $A$ i $B$ nazywamy obiekt $P$ i morfizmy $\morpharr {\pi_A}PA$ oraz $\morpharr {\pi_B}PB$ takie, że jeśli $X$ jest dowolnym obiektem\footnote{Możemy o tym myśleć jak o konkurencie do miana produktu.} oraz $\morpharr {f_A}XA$ i $\morpharr {f_B}XB$, to \emph{istnieje dokładnie jeden} morfizm $\morph fXP$ taki, że $f_A=\pi_A\circ f$ oraz $f_B=\pi_B\circ f$.
  Jest to przedstawione na poniższym diagramie:
  %
  \begin{center}
  \begin{tikzcd}
    & X \arrow[rd, "f_B", bend left] \arrow[ld, "f_A", swap, bend right] \arrow[d, dotted, "! f"] &\\
    A & \arrow[l, "\pi_A", swap] P \arrow[r, "\pi_B"] & B
  \end{tikzcd}
  \end{center}
  %
  Często będziemy oznaczać obiekt produktu $P$ przez $A\times B$.
\end{defn}

\begin{exmp}
	Czym są produkty w $\Set$? Twierdzę, że jednym\footnote{Być może jest więcej, podobnie jak obiektów końcowych... Ciekawe czy wszystkie okażą się izomorficzne?} z dobrych produktów $A$ i $B$ jest $$(A\times B, \morpharr{\pi_A}{A\times B}A, \morpharr{\pi_B}{A\times B}B)$$
  czyli iloczyn kartezjański wraz z rzutami $\actsas{\pi_A}{(a, b)}a$ i analogicznie $\actsas{\pi_B}{(a, b)}b$.

  Weźmy teraz konkurenta do miana produktu - dowolny zbiór $X$ i funkcje $\morph {f_A}XA$ i $\morph {f_B}XB$. Zdefiniujmy:
  $$\morph f X {A\times B}, ~~\actsas{f}{x}{(f_A(x), f_B(x))}$$
  Wówczas:
	$$f_A(x) = \pi_A(f_A(x), f_B(x)) = \pi_A(f(x)) = \pi_A\circ f(x),$$
	czyli $f_A = \pi_A\circ f$ i analogicznie $f_B = \pi_B\circ f$.

  Pozostaje wykazać unikatowość $f$ - weźmy dowolną funkcję $\morph gX{A\times B}$ taką, że $f_A = \pi_A\circ g$ oraz $f_B = \pi_B \circ g$.
  Rozpatrzmy dowolne $x$ i napiszmy dla niego $g(x) = (y, z)$. Wówczas $y = (\pi_A\circ g)(x) = f_A(x)$ i analogicznie $z=f_B(x)$. Czyli $g=f$.
\end{exmp}

\begin{remk}
  Zupełnie dobrymi produktami zbiorów $A$ i $B$ są też:
  \begin{itemize}
    \item $B\times A$ z morfizmami $B\times A \ni (b, a)\mapsto a\in A$ i analogicznym $(b, a)\mapsto b$,
    \item $A\times B\times \{1\}$ z morfizmami $(a, b, 1)\mapsto a$ i $(a, b, 1)\mapsto b$,
    \item dowolny zbiór bijektywny z $A\times B$ gdy wyposaży się go w odpowiednie morfizmy.
  \end{itemize}
\end{remk}

\begin{thm}
  \label{thm:uniqueprod}
  Rozpatrzmy dwa obiekty $A$ i $B$ i przypuśćmy, że istnieje produkt $(P, \pi_A, \pi_B)$. Wówczas:
  \begin{enumerate}
    \item jeśli $(T, p_A, p_B)$ jest produktem, to \emph{istnieje unikatowy izomorfizm} $\morph iTP$ taki, że $p_A = \pi_A\circ i$ oraz $p_B = \pi_B\circ i$,
    \item jeśli mamy izomorfizm $\isomorph fQP$ to $(Q, \pi_A \circ f, \pi_B\circ f)$ też jest produktem,
    \item jeśli $A'\simeq A$ oraz $B'\simeq B$, to produkty $A' \times B'$ i $A\times B$ są izomorficzne.
  \end{enumerate}
\end{thm}

\begin{prof}
  Druga i trzecia część są proste (szczególnie jeśli zrobiło się ćwiczenie \ref{exc:izomorficzne_obiekty}), więc pokażemy tylko dowód pierwszej części:
  \begin{enumerate}
    \item weźmy produkt $(P, \pi_A, \pi_B)$ i potraktujmy jako jego konkurenta $(T, p_A, p_B)$. Mamy unikatowy morfizm $\morph iTP$ taki, że $p_A = \pi_A\circ i$ oraz $p_B = \pi_B\circ i$,
    \begin{center}
    \begin{tikzcd}
      & T \arrow[rd, "p_B", bend left] \arrow[ld, "p_A", swap, bend right] \arrow[d, dotted, "! i"] &\\
      A & \arrow[l, "\pi_A", swap] P \arrow[r, "\pi_B"] & B
    \end{tikzcd}
    \end{center}
    \item teraz weźmy $(T, p_A, p_B)$ jako produkt oraz $(P, \pi_A, \pi_B)$ jako jego konkurencję. Mamy unikatowy morfizm $\morph jPT$ taki, że $\pi_A = p_A\circ j$ oraz $\pi_B = p_B\circ j$,
    \begin{center}
    \begin{tikzcd}
      & P \arrow[rd, "\pi_B", bend left] \arrow[ld, "\pi_A", swap, bend right] \arrow[d, dotted, "! j"] &\\
      A & \arrow[l, "p_A", swap] T \arrow[r, "p_B"] & B
    \end{tikzcd}
    \end{center}
    \item teraz weźmy $(P, \pi_A, \pi_B)$ zarówno jako produkt i współzawodnika. Istnieje unikatowy morfizm $\morph mPP$ taki, że $\pi_A = \pi_A\circ m$ oraz $\pi_B=\pi_B\circ m$.
    \begin{center}
    \begin{tikzcd}
      & P \arrow[rd, "\pi_B", bend left] \arrow[ld, "\pi_A", swap, bend right] \arrow[d, dotted, "! m"] &\\
      A & \arrow[l, "\pi_A", swap] P \arrow[r, "\pi_B"] & B
    \end{tikzcd}
    \end{center}
    \item nietrudno zauważyć, że za $m$ możemy wstawić $\id P$,
    \item teraz popatrzmy na morfizm $i\circ j$. Mamy $\pi_A \circ i\circ j = p_A\circ j = \pi_A$ i~analogicznie $\pi_B\circ i\circ j = \pi_B$. Czyli za $m$ możemy wstawić $i\circ j$,
    \item $m$ jest unikatowe! Czyli $i\circ j = \id P$. Analogicznie $j\circ i=\id T$, czyli $i$ jest izomorfizmem. A skoro jest unikatową strzałką taką, że $p_A=\pi_A\circ i$ oraz $p_B=\pi_B\circ i$, to $i$ jest unikatowym izomorfizmem o zadanej własności.
  \end{enumerate}
\end{prof}

\begin{exc}
  Udowodnij drugą i trzecią część twierdzenia \ref{thm:uniqueprod}.
\end{exc}

\begin{exc}
  Czym jest produkt dwóch grup? A przestrzeni topologicznych?
\end{exc}

\begin{exc}
  Podaj przykład pokazujący, że produkt nie zawsze istnieje.
\end{exc}

\begin{exc}[Pozornie inna definicja produktu]
  \label{exc:produkt_poczatkowy}
  {
  \newcommand{\Prod}{\text{Prod}(A, B)}
  Rozpatrzmy obiekty $A$ i $B$ kategorii $\C$. Definiujemy kategorię ,,obiektów produktopodobnych'' $\Prod$ w następujący sposób:
  \begin{itemize}
    \item obiektami są trójki $(X, f, g)$ gdzie $X$ jest obiektem $\C$, a $\morpharr fXA$ i~$\morpharr gXB$ są morfizmami $\C$,
    \item morfizmem między obiektami $(X, f, g)$ oraz $(Y, p, q)$ będziemy nazywać morfizm $\morpharr mXY$ taki, że $f=p\circ m$ oraz $g=q\circ m$.
  \end{itemize}

  \begin{enumerate}
    \item Po pierwsze upewnij się, że to jest kategoria. (Czym jest składanie morfizmów? Czym są morfizmy identycznościowe? Warto narysować diagram.)
    \item Pokaż, że dowolny produkt $(A\times B, \pi_A, \pi_B)$ jest obiektem końcowym w $\Prod$.
  \end{enumerate}
  Innymi słowy \emph{produkty to obiekty końcowe} (pewnego rodzaju).
  }
\end{exc}

\begin{exc}
  Zdefiniuj produkt dowolnej rodziny obiektów $(A_i)_{i\in I}$.
\end{exc}

\begin{exc}[Zabawy ze zbiorami częściowo uporządkowanymi]
  \begin{enumerate}
    \item Pokaż, że w $\mathbb Z^+$ uporządkowanym przez podzielność (przykład \ref{exmp:dzielenie}) produktem liczb $n$ i $m$ jest $\mathrm{nwd}(n, m)$.
    \item Niech $\mathcal P(S)$ będzie zbiorem potęgowym $S$. Pokaż, że produktem zbiorów $A_1, A_2, \dots, A_n\subseteq S$ jest $A_1\cap A_2\cap \dots \cap A_n$.
    \item Pokaż ogólniejszy fakt - w zbiorze częściowo uporządkowanym $S$, produktem elementów w podzbiorze $X\subseteq X$ jest infimum $X$ (jeśli istnieje). (Element $\inf X$ jest zdefiniowany jako największe ograniczenie dolne $X$, to znaczy $\inf X\le x$ dla wszystkich $x\in X$ oraz jeśli $i\le x$ dla wszystkich $x\in X$, to $i\le \inf X$.)
  \end{enumerate}
\end{exc}

\begin{remk}[O łączności]
  Nie jest trudno wykazać (choć to dość żmudne), że jeśli $(A\times B, \pi_A, \pi_B)$ jest produktem $A$ i $B$, a $((A\times B)\times C,\, p_{A\times B},\, p_C)$ jest produktem $A\times B$ i~$C$, to obiekt $(A\times B)\times C$ wraz z morfizmami $\pi_A\circ p_{A\times B},\, \pi_B\circ p_{A\times B},\, p_C$
  jest produktem $A$, $B$ i $C$. Ma to dwie istotne konsekwencje:
  \begin{enumerate}
    \item jeśli wiemy, że w danej kategorii istnieje produkt każdych dwóch obiektów, to istnieje też produkt każdej ich skończonej liczby,
    \item chociaż często produkt nie jest łączny (np. w teorii mnogości $(A\times B)\times C\neq A\times (B\times C)$), to jednak istnieją unikatowe ,,ładne'' izomorfizmy, utożsamiające różne produkty (jak $((a, b), c)\mapsto (a, (b, c))$). Za pojęciem ,,obiekty są prawie takie same'' stoją izomorfizmy naturalne (pomysł \ref{idea:prawie_takie_same}).
  \end{enumerate}
\end{remk}

\subsection{Koprodukt}
\begin{defn}
  \emph{Koprodukt} jest pojęciem dualnym do produktu, to znaczy - weźmy obiekty $A$ i $B$. Koproduktem nazywamy obiekt $A+B$ oraz morfizmy $\morph{\sigma_A} A{A+B}$ i $\morph{\sigma_B}B{A+B}$, takie, że jeśli $(K, \morpharr {k_A}AK, \morpharr {k_B}BK)$, to istnieje unikatowy morfizm $\morpharr k{A+B}K$ taki, że $k_A = k \circ \sigma_A$ i $k_B = k\circ \sigma_B$.
\end{defn}

\begin{exc}[Własności koproduktu]
  Znając własności produktu i odwracając strzałki, dostajemy analogiczne własności koproduktu:
  \begin{enumerate}
    \item narysuj diagram definiujący koprodukt (wystarczy odwrócić strzałki w~definicji produktu),
    \item zauważ, że jeśli mamy dwa koprodukty $(A+B, \sigma_A, \sigma_B)$ oraz $(K, k_A, k_B)$, to istnieje unikatowy izomorfizm $\morph i{A+B}K$ taki, że $k_A=i\circ \sigma_A$ oraz $k_B=i\circ\sigma_B$,
    \item zinterpretuj koprodukt jako obiekt początkowy w jakiejś kategorii (pomocne może być ćwiczenie \ref{exc:produkt_poczatkowy}).
  \end{enumerate}
\end{exc}

\begin{exmp}
  Koproduktem dwóch zbiorów $A$ i $B$ (lub przestrzeni topologicznych) jest ich suma rozłączna $A\coprod B$, wraz z inkluzjami $A\hookrightarrow A\coprod B$, $B\hookrightarrow A\coprod B$.
\end{exmp}

\begin{exmp}
  Koproduktem dwóch przestrzeni wektorowych (ogólniej - modułów, czyli także grup abelowych) jest ich suma prosta. Jako, że koprodukt i~produkt w tym wypadku są tym samym, nazywamy je czasami \emph{biproduktem}. Natomiast koprodukt nieskończonej rodziny przestrzeni wektorowych (czy modułów) już nie jest tym samym co produkt.
\end{exmp}

\begin{exmp}
  Koproduktem dwóch grup jest iloczyn wolny (ta konstrukcja nie jest podstawowa, mogła być pominięta na kursie algebry).
\end{exmp}

\begin{exc}
  Czym jest koprodukt w $\mathbb Z^+$ uporządkowanym przez podzielność? A w zbiorze potęgowym uporządkowanym przez inkluzję?
\end{exc}

\subsection{Ekwalizator}
\begin{defn}
  Rozważmy równoległe strzałki \begin{tikzcd} X \ar[r, "f", shift left=.75ex] \ar[r, "g", shift right=.75ex, swap] & Y \end{tikzcd} (to jest - dwie strzałki o wspólnej dziedzinie i przeciwdziedzinie. Niekoniecznie $f=g$, więc diagram nie jest do końca przemienny...). \emph{Ekwalizatorem} nazywamy obiekt $E$ i morfizm $\morph eEX$ taki, że:
  \begin{itemize}
    \item $f\circ e = g\circ e$,
    \item jeśli $Q$ i $\morph qEX$ spełniają warunek $f\circ q = g\circ q$, to istnieje unikatowy morfizm $\morph kQE$ taki, że $q=e\circ k$.
  \end{itemize}
  %
  Definicję tę przedstawia poniższy diagram prawie (nie wymagamy $f=g$) przemienny:
  \begin{center}
  \begin{tikzcd}
    E \ar[r, "e"] & X \ar[r, "f", shift left=.75ex] \ar[r, "g", shift right=.75ex, swap] & Y\\
    Q \ar[ru, "q"] \ar[u, "! k", dotted]
  \end{tikzcd}
  \end{center}
\end{defn}

\begin{exc}
  Zinterpretuj stwierdzenie ,,ekwalizator jest unikatowy z dokładnością do unikatowego izomorfizmu''.
\end{exc}

\begin{exc}[Ale dlaczego ,,ekwalizator''?]
  Pokaż, że w $\Set$ ekwalizatorem \begin{tikzcd} X \ar[r, "f", shift left=.75ex] \ar[r, "g", shift right=.75ex, swap] & Y \end{tikzcd} jest zbiór:
    $$E = \{e\in X : f(x) = g(x) \}$$
  wraz z inkluzją $E\hookrightarrow X$.
\end{exc}

\begin{exc}
  Wybierz swoją ulubioną kategorię spośród $\Vect k, \Mod R, \Ab$ i pokaż, że ekwalizatorem morfizmów \begin{tikzcd} X \ar[r, "f", shift left=.75ex] \ar[r, "g", shift right=.75ex, swap] & Y \end{tikzcd} jest jądro ich różnicy (to jest $\ker (f-g)$) wraz z inkluzją.
\end{exc}

\begin{remk}[\starred]
    W $\Mod R$ (czyli też $\Vect k$ i $\Ab$) istnieje obiekt zerowy $0$. Czyli istnieją morfizmy $X\to 0$ i $0\to Y$. Składając je otrzymujemy morfizm zerowy $\morpharr 0 X Y$. W ten sposób otrzymujemy kategoryjną własność jądra $f$ jako ekwalizatora \begin{tikzcd} X \ar[r, "f", shift left=.75ex] \ar[r, "0", shift right=.75ex, swap] & Y \end{tikzcd}.
\end{remk}

\begin{remk}[\starred]
  Ekwalizator (i powyższa interpretacja jako jądro różnicy) są przydatne do zdefiniowania tzw. snopów.
\end{remk}

\begin{exc}
  Zdefiniuj obiekt dualny - koekwalizator. (Jest rzadziej spotykany - w $\Set$ odpowiada dzieleniu przez pewną relację równoważności, a w $\Mod R$ staje się \emph{kojądrem} przekształcenia.)
\end{exc}

% \subsection{Granice i kogranice}
% \begin{idea}
%   W powyższych przykładach powtarza się motyw:
%   \begin{itemize}
%     \item rysujemy diagram składający się z obiektów i strzałek
%     \item szukamy obiektu i strzałek (po jednej dla obiektu), które są ,,uniwersalne'' w pewnym sensie - mając innego kandydata na konstrukcję, jesteśmy w stanie znaleźć unikatowy morfizm między nimi.
%   \end{itemize}
%   Ta idea prowadzi nas do ogólnej definicji granicy (i konstrukcji dualnej - kogranicy).
% \end{idea}
%
% \begin{exmp}
%   Poznaliśmy następujące granice:
%   \begin{enumerate}
%     \item diagram $A~B$ (brak morfizmów) -
%     \item diagram pusty $\varnothing$ - szukamy obiektu $P$
%   \end{enumerate}
% \end{exmp}

\begin{remk}
  Konstrukcje, które wykonywaliśmy nazywane są konstrukcjami uniwersalnymi - zadajemy za pomocą diagramu pewną własność i otrzymujemy obiekt (i morfizmy) odpowiadające obiektowi końcowemu (produkt, ekwalizator) pewnej kategorii lub początkowemu (koprodukt, koekwalizator). W przypadku obiektów końcowych konstrukcje takie nazywamy \emph{granicami}, w przypadku początkowych - \emph{kogranicami}. Dają one zunifikowany pogląd na wiele konstrukcji popularnych w algebrze czy geometrii.
\end{remk}

 \section{Funktory}
\begin{idea}
  Tak jak morfizmy są odwzorowaniami między obiektami, tak \emph{funktory} są \emph{odwzorowaniami między kategoriami}. Pozwala to na przenoszenie twierdzeń i problemów z jednej kategorii do innej.
\end{idea}

\begin{defn}[Funktor kowariantny]
  Niech $\C$ i $\D$ będą kategoriami. \emph{Funktorem kowariantnym} $\morph F\C\D$ nazywamy przyporządkowanie o własnościach:
  \begin{enumerate}
    \item Dla każdego obiektu $A\in \Ob\C$ mamy pewien $F(A)\in \Ob\D$,
    \item Dla każdego morfizmu $\morpharr fAB$, mamy morfizm $\morpharr {F(f)}{F(A)}{F(B)}$.
  \end{enumerate}
 Przy czym wymagamy aby:
  \begin{enumerate}
    \item $F(g\circ f) = F(g)\circ F(f)$ (złożenia są zachowywane),
    \item $F(\id A) = \id{F(A)}$ (identyczności są zachowywane).
  \end{enumerate}
\end{defn}

\begin{exmp}[Funktor stały]
  Niech $D\in \Ob \D$ będzie ustalonym obiektem. Definiujemy:
  \begin{enumerate}
    \item $S(A)=D$
    \item $S(\morpharr fAB) = \id D$
  \end{enumerate}
  Sprawdźmy wszystkie własności:
  \begin{enumerate}
    \item Strzałka $\morph fAB$ przechodzi na strzałkę $\morph{\id D}{S(A)}{S(B)}$,
    \item Oczywiście $S(\id A)=\id {D} = \id {S(A)}$,
    \item $S(g)\circ S(f) = \id D\circ \id D = \id D = S(g\circ f)$.
  \end{enumerate}
  Czyli jest to funktor.
\end{exmp}

\begin{exmp}[Funktor identycznościowy]
  Funktor identycznościowy $\morph {\id \C}\C \C$ to następujące przekształcenie:
  \begin{enumerate}
    \item $\id \C(A) = A$,
    \item $\id \C(\morpharr fAB) = \morpharr fAB$
  \end{enumerate}
  Nietrudno zauważyć, że zachowuje złożenia i identyczności.
\end{exmp}

\begin{exc}
  \label{exc:group_forgetful}
  Pokaż, że odwzorowanie ,,zapominalskie'' $\morph U\Grp \Set$, które:
  \begin{enumerate}
    \item Grupie $G$ przyporządkowuje jej nośnik\footnote{Nośnik to zbiór elementów -- formalnie grupa $G$ to pewien zbiór $U(G)$ i działanie $\circ$ o odpowiednich własnościach.} $U(G)$,
    \item Homomomorfizmowi $G\to H$ przyporządkowuje siebie $U(f)=f$ (homomomorfizmy to przecież funkcje),
  \end{enumerate}
  jest funktorem\footnote{Aby być w pełni poprawnym, to należałoby ,,rozpakować'' homomorfizm z trójki $(G, f, H)$ i ,,zapakować'' go w zbiory, ale nie przejmowałbym się tym szczegółem zanadto. Analogicznym szczegółem jest to, że piszemy $G$ zamiast $(G, \circ)$, bo teoriomnogościowo grupa to zbiór \emph{z działaniem}.}.
\end{exc}

\begin{exc}
  Pokaż, że odwzorowanie $\morph{\mathcal P}\Set\Set$ jest funktorem, gdzie:
  \begin{itemize}
    \item $\mathcal P(A) = \text{zbiór potęgowy $A$}$,
    \item $\mathcal P(f) = \mathrm{Im}\,f = f(A)\subseteq B$, jeśli $\morph fAB$.
  \end{itemize}
\end{exc}

\begin{exmp}[\starred Iloczyn tensorowy]
  Jeśli $V$ jest ustaloną przestrzenią wektorową nad ciałem $k$, to odwzorowanie $W\mapsto W\otimes_k V$, $f\mapsto f\otimes \id V$ jest funktorem $\morph{\bullet \otimes_k V}{\Vect k}{\Vect k}$.
\end{exmp}

\begin{exmp}[\starred Dla topologów algebraicznych]
  W topologii algebraicznej bierzemy przestrzeń topologiczną $X$ oraz punkt $x_0$ i przypisujemy tzw. grupę fundamentalną $\pi_1(X, x_0)$ w punkcie $x_0$. Funkcjom ciągłym przenoszącym wyróżniony punkt na inny wyróżniony punkt, mamy ponadto odpowiadający homomomorfizm grup fundamentalnych, zachowujący złożenia i identyczności. Czyli grupa fundamentalna to tak naprawdę funktor $\morph{\pi_1}{\Top_*}\Grp$ (z kategorii przestrzeni topologicznych z wyróżnionym punktem do kategorii grup).
\end{exmp}

% \begin{exmp}[\starred Dla geometrów różniczkowych]
%   Funktor styczny przypisuje rozmaitości $M$ jej wiązkę styczną $TM$. Funkcji gładkiej $\morph fMN$ przypisuje odwzorowanie styczne: $Tf(v_p) = df|_pv_p\in T_{f(p)}N$.
%   Pochodną funkcji identycznościowej jest oczywiście odwzorowanie identycznościowe --- $(T\id M)(v_p) = d(\id M)|_pv_p = \id {T_pM} v_p$. Podobnie twierdzenie o pochodnej funkcji złożonej ---$T(g\circ f)v_p = d(g\circ f)|_p v_p = dg|_{f(p)}\circ df|_p v_p = (Tg\circ Tf)v_p$ --- pokazuje, że złożenie też się dobrze przenosi.
% \end{exmp}

% \begin{exmp}[\starred Dla koneserów teorii Liego]
%   Jeśli $G$ jest grupą Liego, to możemy przyporządkować jej algebrę Liego $\mathfrak g$ (przestrzeń lewoniezmienniczych pól wektorowych, izomorficzną z przestrzenią styczną obliczoną w identyczności $e\in G$). Ponadto jeśli $\morph \varphi GH$ jest homomomorfizmem grup Liego, to pochodna $d\varphi|_e$ jest homomorfizmem algebr Liego $\mathfrak g\to \mathfrak h$. Nietrudno pokazać, że odwzorowanie to przenosi identyczności na identyczności i zachowuje złożenia.
% \end{exmp}

\begin{remk}
  Funktor \emph{nie musi} ,,dobrze'' przenosić konstrukcji uniwersalnych.

  Rozważmy ,,zapominalski'' funktor $\morph U{\Vect k} \Set$ zamieniający przestrzenie wektorowe na ich nośniki (zbiory wektorów). Jeśli $V$ i $W$ są przestrzeniami wektorowymi, to nośnikiem ich koproduktu (sumy) jest iloczyn kartezjański: $U(V+W) = U(V)\times U(W)$, a koproduktem zbiorów $U(V)$ i~$U(W)$ jest suma rozłączna $U(V)\coprod U(W)$, czyli coś zupełnie innego!
\end{remk}

\begin{exc}
  \label{exc:izo_preserved}
  Pokaż, że funktory zachowują izomorfizmy -- jeśli $\morph fAB$ jest izomorfizmem, to $\morph {F(f)}{F(A)}{F(B)}$ też jest izomorfizmem.
\end{exc}

\begin{remk}
  Jeśli chcemy wykazać, że obiekty $A$ i $B$ nie są izomorficzne, wystarczy znaleźć funktor $F$ taki, że $F(A)$ i $F(B)$ \emph{nie} są izomorficzne. Grupa fundamentalna czy teorie homologii to właśnie sposoby rozróżniania obiektow topologicznych przez zamianę ich na obiekty algebraiczne.
\end{remk}

\begin{defn}[Funktor kontrawariantny]
  \emph{Funktorem kontrawariantnym} $\morph F\C\D$ nazywamy funktor kowariantny $\morph F {\op \C}{\D}$. Innymi słowy:
  \begin{enumerate}
    \item Dla każdego obiektu $A\in \Ob\C$ mamy pewien $F(A)\in \Ob\D$,
    \item Dla morfizmu $\morpharr fAB$, mamy morfizm $\morpharr {F(f)}{F(B)}{F(A)}$.
  \end{enumerate}
 Przy czym wymagamy aby:
  \begin{enumerate}
    \item $F(g\circ f) = F(f)\circ F(g)$ (złożenia są odwracane),
    \item $F(\id A) = \id{F(A)}$ (identyczności są zachowywane).
  \end{enumerate}
\end{defn}

\begin{exc}[\starred Dla topologów]
  Pokaż, że odwzorowanie $\morph{\mathcal P}\Set\Set$ jest funktorem kontrawariantnym, gdzie:
  \begin{itemize}
    \item $\mathcal P(A) = \text{zbiór potęgowy $A$}$,
    \item $\mathcal P(f) = f^{-1}(A) \subseteq B$, jeśli $\morph fAB$.
  \end{itemize}
\end{exc}

\begin{exc}[$\mathrm{Hom}$-funktory]
  Rozważmy kategorię $\C$ i obiekt $A\in \Ob \C$. Stwórzmy odwzorowanie $\morph {\Mor A \bullet} \C \Set$ dane wzorem:
  \begin{itemize}
    \item $\Mor A \bullet(B) = \Mor AB$ (i tak też będziemy pisać --- kropeczka ma symbolizować puste miejsce na argument),
    \item $\Mor A f = f_*$, gdzie jeśli $\morph fB{B'}$, to $\morph{f_*}{\Mor AB}{\Mor A{B'}}$ jest dane wzorem $f_*(\varphi) = f\circ \varphi$.
  \end{itemize}
  W ramach zadania:
  \begin{enumerate}
    \item Pokaż, że jest to funktor kowariantny,
    \item Zdefiniuj \emph{kontrawariantny} funktor $\Mor \bullet A$. (Podpowiedź: już się z~nim spotkaliśmy --- możesz motywować się ćwiczeniem \ref{exc:izomorficzne_obiekty}.)
  \end{enumerate}
  %
  Te funktory są szczególnie ważne w algebrze homologicznej oraz w ważnym rezultacie teorii kategorii -- lemacie Yonedy.
\end{exc}

% \begin{exmp}[\starred Spektrum pierścienia\cite{AtiyahMacdonald}]
% {
%   \newcommand{\CRing}{\textbf{CRing}}
%   \newcommand{\Spec}[1]{\mathrm{Spec}\,#1}
%   W tym przykładzie zajmujemy się $\CRing$, czyli kategorią pierścieni przemiennych z jedynką.
%   Niech $A$ będzie takowym pierścieniem. Możemy mu przypisać zbiór ideałów pierwszych $\Spec A$. Jeśli $\mathfrak p\subseteq B$ jest ideałem pierwszym i $\morph \varphi AB$ jest homomorfizmem, to $\varphi^{-1}(\mathfrak p)$ jest ideałem pierwszym w $A$. Można więc zdefiniować funkcję $\morph{\varphi^*}{\Spec B}{\Spec A}$. Można pokazać, że tak zdefiniowane odwzorowanie jest funktorem kontrawariantnym z $\CRing$ do $\Set$. (A tak naprawdę to w~$\Top$, gdy wyposaży się spektrum w topologię Zariskiego.)
% }
% \end{exmp}

 \section{Transformacje naturalne}
\begin{idea}
  Tak jak funktory to odwzorowania między kategoriami, tak \emph{transformacje naturalne} są odwzorowaniami między funktorami. Innymi słowy mając zadane kategorie $\C$ i $\D$:
  \begin{itemize}
    \item utworzymy\footnote{Modulo problemy teoriomnogościowe - często wymaga się by dziedzina była mała, to znaczy kolekcje strzałek i obiektów były zbiorami.} kolekcję funktorów kowariantnych $\morph F\C\D$,
    \item mając dwa funktory z tej kolekcji (równoległe) $F$ i $G$ określimy odwzorowanie między nimi.
  \end{itemize}
\end{idea}

\begin{defn}
  Niech $\C$ oraz $\D$ będą kategoriami. Przez $\D^\C$ będziemy rozumieć kolekcję funktorów kowariantnych z $\C$ do $\D$.
  Niech $\morph F\C\D$ oraz $\morph G\C\D$ będą takimi funktorami. \emph{Transformacją naturalną} z $F$ do $G$ nazywamy rodzinę morfizmów $\morph{\eta_X}{F(X)}{G(X)}$, gdzie $X\in \Ob\C$ taką, że dla każdego morfizmu $\morpharr fXY$ w $\C$ następujący diagram kategorii $\D$ jest przemienny:
  \begin{center}
    \begin{tikzcd}
      F(X) \ar[r, "F(f)"] \ar[d, "\eta_X"] & F(Y) \ar[d, "\eta_Y"]\\
      G(X) \ar[r, "G(f)"]                 & G(Y)
    \end{tikzcd}
  \end{center}
  %
  Często piszemy $\nattran \eta FG$ oraz przedstawiamy to na obrazku jako:
  \begin{center}
  \begin{tikzcd}[column sep=huge]
    \C
    \arrow[bend left=30]{r}[name=U,label=above:$F$]{}
    \arrow[bend right=30]{r}[name=D,label=below:$G$]{} &
    \D
    \arrow[shorten <=4pt,shorten >=2pt,Rightarrow,to path={(U) -- node[label=right:$\eta$] {} (D)}]{}
  \end{tikzcd}
  \end{center}
\end{defn}

\begin{exmp}
  \label{exmp:transf_ident}
  Dla każdego funktora $F$ istnieje transformacja naturalna $\nattran{\id F}FF$ o składowych $(\id F)_X=\id X$.
\end{exmp}

\begin{remk}
  Możemy też rozpatrywać transformację naturalną między funktorami kontrawariantnymi - jako, że to tak naprawdę funktory z $\op \C$ do $\D$, to wystarczy odwrócić poziome strzałki na diagramie.
\end{remk}

\begin{exc}[Składanie transformacji naturalnych]
  Transformacje naturalne mają być morfizmami kategorii $\D^\C$, potrzebujemy więc określić ich złożenie.

  Niech $F, G, H$ będą funktorami kowariantnymi z $\C$ do $\D$ i niech $\nattran \eta FG$ i $\nattran \gamma GH$ będą transformacjami naturalnymi. Pokaż, że rodzina morfizmów $(\gamma\circ \eta)_X = \gamma_X\circ \eta_X$ jest transformacją naturalną $F\Rightarrow H$.
  \begin{center}
    \begin{tikzcd}[column sep=huge]
      \C
      \arrow[bend left=50]{r}[name=U,label=above:$F$]{}
      \arrow{r}[near start, label=below:$G$]{}[name=M]{}
      \arrow[bend right=50]{r}[name=D,label=below:$H$]{}
      &
      \D
      \arrow[shorten <=4pt,shorten >=0pt,Rightarrow,to path={(U) -- node[label=right:$\eta$] {} (M)}]{}
      \arrow[shorten <=2pt,shorten >=0pt,Rightarrow,to path={(M) -- node[label=right:$\gamma$] {} (D)}]{}
    \end{tikzcd}
  \end{center}
  %
  (Należy sprawdzić przemienność poniższego diagramu. W razie problemów warto wrócić do ćwiczenia \ref{exc:setstardlabystrzakow}.)
  \begin{center}
    \begin{tikzcd}
      F(X) \ar[r, "F(f)"] \ar[d, "\gamma_X\circ \eta_X"] & F(Y) \ar[d, "\gamma_Y\circ \eta_Y"]\\
      H(X) \ar[r, "H(f)"]                 &  H(Y)
    \end{tikzcd}
  \end{center}
\end{exc}

\begin{exc}
  Dokończ uzasadniać dlaczego $\D^\C$ jest kategorią, to znaczy:
  \begin{itemize}
    \item pokaż łączność złożeń,
    \item pokaż istnienie morfizmów identycznościowych (przypomnij sobie przykład \ref{exmp:transf_ident}).
  \end{itemize}
\end{exc}

\begin{remk}
  Rozważmy (ignorując problemy teoriomnogościowe rodzaju zbioru wszystkich zbiorów) kategorię w której obiektami są kategorie, a morfizmami - funktory. Mamy też 2-morfizmy (transformacje naturalne), które są strzałkami między 1-morfizmami (funktorami). Mamy więc bogatszą strukturę niż zwykłej kategorii, nazywaną \emph{2-kategorią}. Można też rozważać jeszcze bardziej rozbudowane twory, jak 3, 4 czy $\infty-$kategorie, czym zajmuje się \emph{wyższa teoria kategorii}.
\end{remk}

\begin{idea}
  \label{idea:prawie_takie_same}
  Zbiory $A\times B$ i $B\times A$ są ,,prawie takie same'' - oczywiście są izomorficzne, co więcej izomorfizm między nimi jest ładny: $(a, b)\mapsto (b, a)$. Podobnie ,,prawie takie same'' są zbiory $(A\times B)\times C$ i $A\times (B\times C)$.

  W teorii kategorii umawiamy się na utożsamianie izomorficznych obiektów \emph{mając zadany izomorfizm między nimi}. Naturalny izomorfizm okazuje się być rodziną izomorfizmów - regułą w jaki sposób należy utożsamiać różne przestrzenie. To znaczy zamiast zadawać jeden izomorfizm między dwoma obiektami, zadajemy naraz izomorfizmy na bardzo wielu parach obiektów!
\end{idea}

\begin{exc}[Naturalny izomorfizm]
  Pokaż, że $\nattran \eta FG$ jest izomorfizmem w $\D^\C$ wtedy i tylko wtedy gdy wszystkie morfizmy $\eta_X$ są izomorfizmami w $\D$. W takim wypadku $\eta$ nazywane jest \emph{naturalnym izomorfizmem}.
\end{exc}

\begin{nott}
  Jeśli $\morph fA{A'}$ oraz $\morph gB{B'}$ są funkcjami, to określamy funkcję $\morph{f\times g}{A\times B}{A'\times B'}$ daną wzorem $(a, b)\mapsto (f(a), g(b))$.
\end{nott}

\begin{exc}
  Rozważmy kategorię $\Set\times \Set$, której obiektami są pary zbiorów $(A, B)$ i morfizmami są pary strzałek $(\morpharr fA {A'}, \morpharr gB{B'})$ oraz dwa równoległe funktory:
  \begin{itemize}
    \item $\morph {P_1}{\Set\times \Set}\Set$, $P(A, B)=A\times B$, $P(f, g) = f\times g$,
    \item $\morph {P_2}{\Set\times \Set}\Set$, $P(A, B)=B\times A$, $P(f, g) = g\times f$.
  \end{itemize}
  Pokaż, że transformacja naturalna $\nattran{\eta_{X, Y}}{P_1}{P_2}$ dana wzorem $\morph{\eta_{X, Y}}{X\times Y}{Y\times X}$, $(x, y)\mapsto (y, x)$ jest \emph{naturalnym izomorfizmem}.
\end{exc}

\subsection{\starred Przestrzeń dwukrotnie dualna}
{
\newcommand{\dual}[1]{{#1}^{*}}
\newcommand{\ddual}[1]{{#1}^{**}}

\begin{idea}
  Przestrzeń wektorowa skończonego wymiaru $V$ jest izomorficzna do swojej przestrzeni dualnej $\dual V$, jednak dla każdej przestrzeni istnieje ,,zupełnie inny'' izomorfizm. Natomiast istnieje rodzina izomorfizmów utożsamiająca $V$ i $\ddual V$ - tak często stosowany, że często się pisze $V=\ddual V$.
\end{idea}

\begin{nott}
  Pracujemy w kategorii przestrzeni wektorowych skończonego wymiaru $\C=\FinVect k$.
\end{nott}

\begin{defn}
  Niech $V\in \Ob \C$. Definiujemy \emph{przestrzeń dualną}:
    $$\dual V = \Mor Vk$$
  Definiując dodawanie i mnożenie przez skalary nadajemy $\dual V$ strukturę przestrzeni wektorowej:
  \begin{align*}
    (\nu+\omega)(v) &:= \nu(v) + \omega(v)\\
    (a\cdot \nu)(v) &:= a\cdot \nu(v)
  \end{align*}
  Elementy $\dual V$ nazywamy \emph{kowektorami} lub \emph{jednoformami}.
\end{defn}

\begin{thm}
  Przestrzenie $V$ i $\dual V$ są izomorficzne.
\end{thm}

\begin{prof}
  Weźmy dowolną bazę $v_1,\, v_2,\, \dots,\, v_n$ przestrzeni $V$. Definiujemy $n$ jednoform wzorami:
  $$\nu_i(a_1v_1 + \dots + a_nv_n) = a_i$$
  Nietrudno zauważyć, że są to odwzorowania liniowe, czyli rzeczywiście $\nu_i$ są jednoformami. Jeśli $a_1\nu_1+\dots + a_n\nu_n=0$, to $(a_1\nu_1+\dots + a_n\nu_n)(v_i)=a_i=0$ czyli formy te są liniowo niezależne.
  Weźmy teraz dowolną funkcję liniową $\omega\in \dual V$. Skoro jest to funkcja liniowa, jest jednoznacznie wyznaczona przez wartości przyjmowane na bazie $\omega(v_1),\,\dots,\,\omega(v_n)$. Zachodzi $\omega = \omega(v_1)\cdot \nu_1+\dots+\omega(v_n)\cdot \nu_n$, co kończy dowód.
\end{prof}

\begin{remk}
  Jeśli wymiar $V$ jest nieskończony, to $\dim \dual V > \dim V$.
\end{remk}

\begin{defn}
  Jeśli $\morph fVW$, to definiujemy \emph{odwzorowanie dualne} $\morph{\dual f}{\dual W}{\dual V}$ dane wzorem:
  $$\dual f(\omega) = \omega \circ f$$
  Nietrudno zauważyć, że rzeczywiście $\dual f(\omega)\in \dual V$ oraz, że $\dual f$ jest odwzorowaniem liniowym.
\end{defn}

\begin{remk}
  Mamy do czynienia z funktorem \emph{kontrawariantnym}: $V\mapsto \dual V$, $f\mapsto \dual f$.
\end{remk}

\begin{defn}
  \emph{Przestrzenią dwukrotnie dualną} $\ddual V$ nazywamy przestrzeń dualną przestrzeni dualnej $(\dual V)^*$. Podobnie dla $\morph fVW$ definiujemy odwzorowanie dwukrotnie dualne $\morph{\ddual f}{\ddual V}{\ddual W}$.
  Jeśli $\bar v\in \ddual V$ oraz $\omega \in \dual W$, wyraża się ono wzorem:
    $$\ddual f(\bar v)(\omega) = \bar v(\dual f(\omega)) = \bar v(\omega\circ f)$$
\end{defn}

\begin{cor}
  Przekształcenie $V\mapsto \ddual V$, $f\mapsto \ddual f$ jest funktorem kowariantnym (jako złożenie funktorów kontrawariantnych).
\end{cor}

\begin{thm}
  Funktor identycznościowy $\id \C$ oraz funktor dwukrotnie dualny są naturalnie izomorficzne.
\end{thm}

\begin{prof}
  Oznaczmy funktor dwukrotnie dualny przez $D$. Potrzebujemy naturalnego izomorfizmu $\nattran \eta {\id \C}D$. Spróbujmy więc:
  $$\morph{\eta_V}{V}{\ddual V},~\eta_V(v)(\nu) := \nu(v)$$
  Nietrudno zauważyć, że $\eta_V$ jest przekształceniem liniowym. Pozostaje sprawdzić:
  \begin{enumerate}
    \item czy $\eta$ jest w ogóle transformacją naturalną?
    \item czy funkcje $\eta_V$ są izomorfizmami?
  \end{enumerate}
  %
  Potrzebujemy sprawdzić czy diagram:
  \begin{center}
  \begin{tikzcd}
    V \ar[r, "f"] \ar[d, "\eta_V", swap]& W \ar[d, "\eta_W"]\\
    \ddual V \ar[r, "\ddual f"] & \ddual W
  \end{tikzcd}
  \end{center}
  jest przemienny. Niech $v\in V$ oraz $\omega\in W^*$. Mamy:
  $$((\eta_W\circ f)(v))(\omega)=\omega(f(v))$$
  $$((\ddual f\circ \eta_V)(v))(\omega) = ( \ddual f(\eta_V(v)))(\omega) = (\eta_V(v))(\omega\circ f) = (\omega\circ f)(v) = \omega(f(v))$$

  Udowodnimy teraz, że $\eta_V$ są izomorfizmami. Niech $\eta_V(v) = 0$, czyli $\nu(v)=0$ dla wszystkich $\nu\in V^*$. Stąd $v=0$ (inaczej możemy uzupełnić $v$ do bazy i rozpatrzyć bazę dualną, której pierwszy wektor da wynik 1 zamiast 0). Jądro $\eta_V$ jest trywialne, a skoro $\dim \ddual V = \dim \dual V = \dim V$, to z twierdzenia o~rzędzie otrzymujemy tezę.
\end{prof}

\begin{remk}
    Pokazaliśmy, że $\eta$ jest naturalnym izomorfizmem. Natomiast funktor przypisujący przestrzeń dualną $V^*$ jest \emph{kontrawariantny}, więc nie może być naturalnie izomorficzny z~identycznością. Tego naturalnego izomorfizmu $\eta$ używa się często niejawnie do identyfikowania $V$ i $\ddual V$ tak, że popularnie pisze się $V=\ddual V$, podobnie jak często identyfikuje się zbiory $(A\times B)\times C$ oraz $A\times (B\times C)$ bez wspominania o tym.

    Ten przykład jest szczególnie ważny z powodów historycznych - zapoczątkował teorię kategorii. To właśnie transformacje naturalne były motywacją do stworzenia funktorów i kategorii.
\end{remk}
}

 \section{Posłowie}
\paragraph{Co mam zapamiętać?}
\begin{enumerate}
  \item Kategoria to kolekcja obiektów powiązanych morfizmami.
  \item Nie interesuje nas \emph{wewnętrzna struktura} obiektu (jak elementy zbioru) -- ale to w jaki sposób oddziałuje z innymi obiektami (jakie morfizmy do niego wchodzą i z niego wychodzą).
  \item Jako, że izomorficzne obiekty oddziałują tak samo, staramy się przymykać oko na różnice między nimi.
  \item Konstrukcje nowych obiektów (jak iloczyn kartezjański czy suma przestrzeni wektorowych) mają swoje uogólnienia, składające się z obiektu i rodziny morfizmów.
  \item Tak jak w kategorii różne obiekty oddziałują poprzez morfizmy, tak różne \emph{kategorie} oddziałują przez funktory. Wiele konstrukcji znanych z algebry czy geometrii to właśnie funktory.
  \item Transformacje naturalne są morfizmami między funktorami. Notacja ,,prawie takie same'' jest naturalnym izomorfizmem pewnych funktorów.
\end{enumerate}

\paragraph{Co mam robić teraz?}
\begin{enumerate}
  \item Widząc nową konstrukcję, zastanów się czy nie jest ona kategoryjna:
    \begin{enumerate}
      \item Czy nie jest charakteryzowana przez pewien diagram i własność uniwersalną?
      \item Czy nie jest funktorialna?
      \item A może jest naturalnym izomorfizmem?
    \end{enumerate}
  \item Zapoznaj się z \emph{granicami, sprzężeniami} i \emph{lematem Yonedy}. Dobrymi książkami dla początkujących są \cite{Awodey, Baez, Leinster}.
  \item Jeśli szukasz zastosowań teorii kategorii w innych dziedzinach, możesz zainteresować się algebrą przemienną \cite{AtiyahMacdonald}, algebrą homologiczną, topologią algebraiczną \cite{May} czy geometrią algebraiczną \cite{Vakil}. Możesz też przeczytać \cite{Rosetta}.
  \item Jeśli lubisz logikę, oprócz \cite{Rosetta, Awodey}, możesz zacząć czytać o teorii \emph{toposów} \cite{Barr, MacLane_Moerdijk}.
  \item Jeśli czujesz się bardzo pewnie, spójrz na legendarną książkę \cite{MacLane} -- autorem jest jeden z twórców teorii kategorii, a treść zaawansowana.
  \item ... i pamiętaj: zawsze możesz sprawdzić nLab \cite{nLab}.
\end{enumerate}

\paragraph{Sugestie i błędy} Będę wdzięczny za informację o znalezionych błędach, sugestiach czy wrażeniach z czytania (np. co było za trudne, a co zbyt łatwe, ile czasu pochłonęły te notatki).

\paragraph{Podziękowania} Chciałbym podziękować Fredericowi Grabowskiemu za przemyślenia co powinno się znaleźć we wprowadzeniu do teorii kategorii; uczestnikom zajęć z teorii kategorii WWW14 za pokazanie mi co jest zrozumiałe, a co nie; oraz Iwonie Kotlarskiej za cenne uwagi dotyczące zarówno wyboru treści jak i stylistyki.

 \begin{thebibliography}{99}
\bibitem{AtiyahMacdonald}
  M. F. Atiyah, I. G MacDonald,
  \emph{Introduction to Commutative Algebra}.
  Addison Wesley, Massachusetts, 1969.
\bibitem{Awodey}
  S. Awodey,
  \emph{Category theory},
  Notatki dostępne pod: \url{http://www.andrew.cmu.edu/course/80-413-713/notes/}.
\bibitem{Baez}
    J. C. Baez,
    \emph{Category Theory Course},
    Notatki dostępne pod:
    \url{http://math.ucr.edu/home/baez/qg-winter2016/CategoryTheoryNotes.pdf}.
\bibitem{Rosetta}
    J. C. Baez, M. Stay
    \emph{Physics, Topology, Logic and Computation: A Rosetta Stone}.
    Notatki dostępne pod: \url{http://math.ucr.edu/home/baez/rosetta.pdf}.
\bibitem{Barr}
  M. Barr, C. Wells,
  \emph{Toposes, Triples and Theories},
  Wydanie dostępne w internecie: \url{http://www.tac.mta.ca/tac/reprints/articles/12/tr12abs.html}.
\bibitem{Leinster}
  T. Leinster,
  \emph{Basic Category Theory},
  Dostępne pod: \url{https://arxiv.org/pdf/1612.09375.pdf}.
\bibitem{rethinking_set_theory}
  T. Leinster,
  \emph{Rethinking Set Theory},
  Dostępne pod: \url{https://arxiv.org/pdf/1212.6543v1.pdf}
\bibitem{MacLane}
  S. Maclane,
  \emph{Categories for the Working Mathematician},
  Springer, 1978.
\bibitem{MacLane_Moerdijk}
  S. MacLane, I. Moerdijk,
  \emph{Sheaves in Geometry and Logic: A First Introduction to Topos Theory},
  Springer, 2012.
\bibitem{May}
  J. P. May,
  \emph{A Concise Course in Algebraic Topology},
  Notatki dostępne pod: \url{https://www.math.uchicago.edu/~may/CONCISE/ConciseRevised.pdf}.
\bibitem{nLab}
  nLab,
  \url{https://ncatlab.org/nlab/show/HomePage}
\bibitem{Vakil}
    R. Vakil,
    \emph{The Rising Sea: Foundations of Algebraic Geometry}.
    Notatki dostępne pod: \url{http://math.stanford.edu/~vakil/216blog/index.html}.
\end{thebibliography}


% \section{Sprzężenia}
\begin{noth}[Przestrzeń wolna]
  Niech $S$ będzie zbiorem, a $k$ ustalonym ciałem. Weźmy funkcję $\morph f Sk$ taką, że $f(v) = 0$ dla prawie wszystkich $v$ (czyli istnieje tylko skończenie wiele wektorów $v$ takich, że $f(v)\neq 0$).

  Nietrudno zauważyć, że funkcje takie tworzą przestrzeń wektorową.
\end{noth}

\begin{exmp}
  Rozważmy kategorie $\Vect k$ oraz $\Set$. Możemy wprowadzić dwa funktory:
  \begin{itemize}
    \item ,,funktor zapominalski'' $\morph{U}{\Vect k}{\Set}$, który przestrzeni wektorowej przyporządkowuje jej nośnik (zbiór wektorów) i nie zmienia funkcji (bo funkcje liniowe to przecież funkcje!),
    \item ,,wolny funktor'' $\morph{F}{\Set}{\Vect k}$. Mając dany zbiór $S$ przekształca go na wolną przestrzeń wektorową.
  \end{itemize}
\end{exmp}

\end{document}
