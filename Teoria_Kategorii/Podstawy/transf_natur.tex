\section{Transformacje naturalne}
\begin{idea}
  Tak jak funktory to odwzorowania między kategoriami, tak \emph{transformacje naturalne} są odwzorowaniami między funktorami. Innymi słowy mając zadane kategorie $\C$ i $\D$:
  \begin{itemize}
    \item utworzymy\footnote{Modulo problemy teoriomnogościowe - często wymaga się by dziedzina była mała, to znaczy kolekcje strzałek i obiektów były zbiorami.} kolekcję funktorów kowariantnych $\morph F\C\D$,
    \item mając dwa funktory z tej kolekcji (równoległe) $F$ i $G$ określimy odwzorowanie między nimi.
  \end{itemize}
\end{idea}

\begin{defn}
  Niech $\C$ oraz $\D$ będą kategoriami. Przez $\D^\C$ będziemy rozumieć kolekcję funktorów kowariantnych z $\C$ do $\D$.
  Niech $\morph F\C\D$ oraz $\morph G\C\D$ będą takimi funktorami. \emph{Transformacją naturalną} z $F$ do $G$ nazywamy rodzinę morfizmów $\morph{\eta_X}{F(X)}{G(X)}$, gdzie $X\in \Ob\C$ taką, że dla każdego morfizmu $\morpharr fXY$ w $\C$ następujący diagram kategorii $\D$ jest przemienny:
  \begin{center}
    \begin{tikzcd}
      F(X) \ar[r, "F(f)"] \ar[d, "\eta_X"] & F(Y) \ar[d, "\eta_Y"]\\
      G(X) \ar[r, "G(f)"]                 & G(Y)
    \end{tikzcd}
  \end{center}
  %
  Często piszemy $\nattran \eta FG$ oraz przedstawiamy to na obrazku jako:
  \begin{center}
  \begin{tikzcd}[column sep=huge]
    \C
    \arrow[bend left=30]{r}[name=U,label=above:$F$]{}
    \arrow[bend right=30]{r}[name=D,label=below:$G$]{} &
    \D
    \arrow[shorten <=4pt,shorten >=2pt,Rightarrow,to path={(U) -- node[label=right:$\eta$] {} (D)}]{}
  \end{tikzcd}
  \end{center}
\end{defn}

\begin{exmp}
  \label{exmp:transf_ident}
  Dla każdego funktora $F$ istnieje transformacja naturalna $\nattran{\id F}FF$ o składowych $(\id F)_X=\id X$.
\end{exmp}

\begin{remk}
  Możemy też rozpatrywać transformację naturalną między funktorami kontrawariantnymi - jako, że to tak naprawdę funktory z $\op \C$ do $\D$, to wystarczy odwrócić poziome strzałki na diagramie.
\end{remk}

\begin{exc}[Składanie transformacji naturalnych]
  Transformacje naturalne mają być morfizmami kategorii $\D^\C$, potrzebujemy więc określić ich złożenie.

  Niech $F, G, H$ będą funktorami kowariantnymi z $\C$ do $\D$ i niech $\nattran \eta FG$ i $\nattran \gamma GH$ będą transformacjami naturalnymi. Pokaż, że rodzina morfizmów $(\gamma\circ \eta)_X = \gamma_X\circ \eta_X$ jest transformacją naturalną $F\Rightarrow H$.
  \begin{center}
    \begin{tikzcd}[column sep=huge]
      \C
      \arrow[bend left=50]{r}[name=U,label=above:$F$]{}
      \arrow{r}[near start, label=below:$G$]{}[name=M]{}
      \arrow[bend right=50]{r}[name=D,label=below:$H$]{}
      &
      \D
      \arrow[shorten <=4pt,shorten >=0pt,Rightarrow,to path={(U) -- node[label=right:$\eta$] {} (M)}]{}
      \arrow[shorten <=2pt,shorten >=0pt,Rightarrow,to path={(M) -- node[label=right:$\gamma$] {} (D)}]{}
    \end{tikzcd}
  \end{center}
  %
  (Należy sprawdzić przemienność poniższego diagramu. W razie problemów warto wrócić do ćwiczenia \ref{exc:setstardlabystrzakow}.)
  \begin{center}
    \begin{tikzcd}
      F(X) \ar[r, "F(f)"] \ar[d, "\gamma_X\circ \eta_X"] & F(Y) \ar[d, "\gamma_Y\circ \eta_Y"]\\
      H(X) \ar[r, "H(f)"]                 &  H(Y)
    \end{tikzcd}
  \end{center}
\end{exc}

\begin{exc}
  Dokończ uzasadniać dlaczego $\D^\C$ jest kategorią, to znaczy:
  \begin{itemize}
    \item pokaż łączność złożeń,
    \item pokaż istnienie morfizmów identycznościowych (przypomnij sobie przykład \ref{exmp:transf_ident}).
  \end{itemize}
\end{exc}

\begin{remk}
  Rozważmy (ignorując problemy teoriomnogościowe rodzaju zbioru wszystkich zbiorów) kategorię w której obiektami są kategorie, a morfizmami - funktory. Mamy też 2-morfizmy (transformacje naturalne), które są strzałkami między 1-morfizmami (funktorami). Mamy więc bogatszą strukturę niż zwykłej kategorii, nazywaną \emph{2-kategorią}. Można też rozważać jeszcze bardziej rozbudowane twory, jak 3, 4 czy $\infty-$kategorie, czym zajmuje się \emph{wyższa teoria kategorii}.
\end{remk}

\begin{idea}
  \label{idea:prawie_takie_same}
  Zbiory $A\times B$ i $B\times A$ są ,,prawie takie same'' - oczywiście są izomorficzne, co więcej izomorfizm między nimi jest ładny: $(a, b)\mapsto (b, a)$. Podobnie ,,prawie takie same'' są zbiory $(A\times B)\times C$ i $A\times (B\times C)$.

  W teorii kategorii umawiamy się na utożsamianie izomorficznych obiektów \emph{mając zadany izomorfizm między nimi}. Naturalny izomorfizm okazuje się być rodziną izomorfizmów - regułą w jaki sposób należy utożsamiać różne przestrzenie. To znaczy zamiast zadawać jeden izomorfizm między dwoma obiektami, zadajemy naraz izomorfizmy na bardzo wielu parach obiektów!
\end{idea}

\begin{exc}[Naturalny izomorfizm]
  Pokaż, że $\nattran \eta FG$ jest izomorfizmem w $\D^\C$ wtedy i tylko wtedy gdy wszystkie morfizmy $\eta_X$ są izomorfizmami w $\D$. W takim wypadku $\eta$ nazywane jest \emph{naturalnym izomorfizmem}.
\end{exc}

\begin{nott}
  Jeśli $\morph fA{A'}$ oraz $\morph gB{B'}$ są funkcjami, to określamy funkcję $\morph{f\times g}{A\times B}{A'\times B'}$ daną wzorem $(a, b)\mapsto (f(a), g(b))$.
\end{nott}

\begin{exc}
  Rozważmy kategorię $\Set\times \Set$, której obiektami są pary zbiorów $(A, B)$ i morfizmami są pary strzałek $(\morpharr fA {A'}, \morpharr gB{B'})$ oraz dwa równoległe funktory:
  \begin{itemize}
    \item $\morph {P_1}{\Set\times \Set}\Set$, $P(A, B)=A\times B$, $P(f, g) = f\times g$,
    \item $\morph {P_2}{\Set\times \Set}\Set$, $P(A, B)=B\times A$, $P(f, g) = g\times f$.
  \end{itemize}
  Pokaż, że transformacja naturalna $\nattran{\eta_{X, Y}}{P_1}{P_2}$ dana wzorem $\morph{\eta_{X, Y}}{X\times Y}{Y\times X}$, $(x, y)\mapsto (y, x)$ jest \emph{naturalnym izomorfizmem}.
\end{exc}

\subsection{\starred Przestrzeń dwukrotnie dualna}
{
\newcommand{\dual}[1]{{#1}^{*}}
\newcommand{\ddual}[1]{{#1}^{**}}

\begin{idea}
  Przestrzeń wektorowa skończonego wymiaru $V$ jest izomorficzna do swojej przestrzeni dualnej $\dual V$, jednak dla każdej przestrzeni istnieje ,,zupełnie inny'' izomorfizm. Natomiast istnieje rodzina izomorfizmów utożsamiająca $V$ i $\ddual V$ - tak często stosowany, że często się pisze $V=\ddual V$.
\end{idea}

\begin{nott}
  Pracujemy w kategorii przestrzeni wektorowych skończonego wymiaru $\C=\FinVect k$.
\end{nott}

\begin{defn}
  Niech $V\in \Ob \C$. Definiujemy \emph{przestrzeń dualną}:
    $$\dual V = \Mor Vk$$
  Definiując dodawanie i mnożenie przez skalary nadajemy $\dual V$ strukturę przestrzeni wektorowej:
  \begin{align*}
    (\nu+\omega)(v) &:= \nu(v) + \omega(v)\\
    (a\cdot \nu)(v) &:= a\cdot \nu(v)
  \end{align*}
  Elementy $\dual V$ nazywamy \emph{kowektorami} lub \emph{jednoformami}.
\end{defn}

\begin{thm}
  Przestrzenie $V$ i $\dual V$ są izomorficzne.
\end{thm}

\begin{prof}
  Weźmy dowolną bazę $v_1,\, v_2,\, \dots,\, v_n$ przestrzeni $V$. Definiujemy $n$ jednoform wzorami:
  $$\nu_i(a_1v_1 + \dots + a_nv_n) = a_i$$
  Nietrudno zauważyć, że są to odwzorowania liniowe, czyli rzeczywiście $\nu_i$ są jednoformami. Jeśli $a_1\nu_1+\dots + a_n\nu_n=0$, to $(a_1\nu_1+\dots + a_n\nu_n)(v_i)=a_i=0$ czyli formy te są liniowo niezależne.
  Weźmy teraz dowolną funkcję liniową $\omega\in \dual V$. Skoro jest to funkcja liniowa, jest jednoznacznie wyznaczona przez wartości przyjmowane na bazie $\omega(v_1),\,\dots,\,\omega(v_n)$. Zachodzi $\omega = \omega(v_1)\cdot \nu_1+\dots+\omega(v_n)\cdot \nu_n$, co kończy dowód.
\end{prof}

\begin{remk}
  Jeśli wymiar $V$ jest nieskończony, to $\dim \dual V > \dim V$.
\end{remk}

\begin{defn}
  Jeśli $\morph fVW$, to definiujemy \emph{odwzorowanie dualne} $\morph{\dual f}{\dual W}{\dual V}$ dane wzorem:
  $$\dual f(\omega) = \omega \circ f$$
  Nietrudno zauważyć, że rzeczywiście $\dual f(\omega)\in \dual V$ oraz, że $\dual f$ jest odwzorowaniem liniowym.
\end{defn}

\begin{remk}
  Mamy do czynienia z funktorem \emph{kontrawariantnym}: $V\mapsto \dual V$, $f\mapsto \dual f$.
\end{remk}

\begin{defn}
  \emph{Przestrzenią dwukrotnie dualną} $\ddual V$ nazywamy przestrzeń dualną przestrzeni dualnej $(\dual V)^*$. Podobnie dla $\morph fVW$ definiujemy odwzorowanie dwukrotnie dualne $\morph{\ddual f}{\ddual V}{\ddual W}$.
  Jeśli $\bar v\in \ddual V$ oraz $\omega \in \dual W$, wyraża się ono wzorem:
    $$\ddual f(\bar v)(\omega) = \bar v(\dual f(\omega)) = \bar v(\omega\circ f)$$
\end{defn}

\begin{cor}
  Przekształcenie $V\mapsto \ddual V$, $f\mapsto \ddual f$ jest funktorem kowariantnym (jako złożenie funktorów kontrawariantnych).
\end{cor}

\begin{thm}
  Funktor identycznościowy $\id \C$ oraz funktor dwukrotnie dualny są naturalnie izomorficzne.
\end{thm}

\begin{prof}
  Oznaczmy funktor dwukrotnie dualny przez $D$. Potrzebujemy naturalnego izomorfizmu $\nattran \eta {\id \C}D$. Spróbujmy więc:
  $$\morph{\eta_V}{V}{\ddual V},~\eta_V(v)(\nu) := \nu(v)$$
  Nietrudno zauważyć, że $\eta_V$ jest przekształceniem liniowym. Pozostaje sprawdzić:
  \begin{enumerate}
    \item czy $\eta$ jest w ogóle transformacją naturalną?
    \item czy funkcje $\eta_V$ są izomorfizmami?
  \end{enumerate}
  %
  Potrzebujemy sprawdzić czy diagram:
  \begin{center}
  \begin{tikzcd}
    V \ar[r, "f"] \ar[d, "\eta_V", swap]& W \ar[d, "\eta_W"]\\
    \ddual V \ar[r, "\ddual f"] & \ddual W
  \end{tikzcd}
  \end{center}
  jest przemienny. Niech $v\in V$ oraz $\omega\in W^*$. Mamy:
  $$((\eta_W\circ f)(v))(\omega)=\omega(f(v))$$
  $$((\ddual f\circ \eta_V)(v))(\omega) = ( \ddual f(\eta_V(v)))(\omega) = (\eta_V(v))(\omega\circ f) = (\omega\circ f)(v) = \omega(f(v))$$

  Udowodnimy teraz, że $\eta_V$ są izomorfizmami. Niech $\eta_V(v) = 0$, czyli $\nu(v)=0$ dla wszystkich $\nu\in V^*$. Stąd $v=0$ (inaczej możemy uzupełnić $v$ do bazy i rozpatrzyć bazę dualną, której pierwszy wektor da wynik 1 zamiast 0). Jądro $\eta_V$ jest trywialne, a skoro $\dim \ddual V = \dim \dual V = \dim V$, to z twierdzenia o~rzędzie otrzymujemy tezę.
\end{prof}

\begin{remk}
    Pokazaliśmy, że $\eta$ jest naturalnym izomorfizmem. Natomiast funktor przypisujący przestrzeń dualną $V^*$ jest \emph{kontrawariantny}, więc nie może być naturalnie izomorficzny z~identycznością. Tego naturalnego izomorfizmu $\eta$ używa się często niejawnie do identyfikowania $V$ i $\ddual V$ tak, że popularnie pisze się $V=\ddual V$, podobnie jak często identyfikuje się zbiory $(A\times B)\times C$ oraz $A\times (B\times C)$ bez wspominania o tym.

    Ten przykład jest szczególnie ważny z powodów historycznych - zapoczątkował teorię kategorii. To właśnie transformacje naturalne były motywacją do stworzenia funktorów i kategorii.
\end{remk}
}
